\documentclass[doktyp=barbeit]{TUBAFarbeiten}

\usepackage[ngerman,english]{babel}

\usepackage{selinput}% Auswahl der Dateikodierung (ansi,latin1,utf8,...)
	\SelectInputMappings{adieresis={ä},germandbls={ß},Euro={€}}% Zeichenzuordnung für selinput.sty
\usepackage[T1]{fontenc}% Einstellung Fontencoding

\usepackage{csquotes}% Einstellung zu Anführungszeichen; wird von biblatex.sty gefordert
\usepackage[backend=biber,style=numeric,sorting=none,sortlocale=de_DE_phonebook]{biblatex}% bessere Literaturverarbeitung
\addbibresource{literaturquellen.bib}

\usepackage{hyperref}
\usepackage[table,figure]{hypcap}

\usepackage{tabularx,diagbox}
\usepackage{float}

\usepackage{amsmath}

\usepackage{listings}
\definecolor{gray}{rgb}{0.95,0.95,0.95}
\definecolor{violet}{rgb}{0.5, 0.0, 0.6}
\lstset{ basicstyle=\ttfamily\small, backgroundcolor=\color{gray} }
\lstset{ numbers=left, numberstyle=\tiny, numbersep=2pt }
\lstset{ breaklines=true, keywordstyle=\color{violet} }

%\usepackage{setspace}% Einstellungen Zeilenabstand
	%\onehalfspacing% Einstellungen Zeilenabstand

\setcounter{secnumdepth}{4}
% \setcounter{tocdepth}{4}

% Worttrennung für nicht bekannte Wörter
\hyphenation{Con-nect-able}
\hyphenation{Con-nec-tions}
\hyphenation{Es-tab-lish-ment}
\hyphenation{Mode}
\hyphenation{Ad-ver-tise-ment}
\hyphenation{Schlüs-sel-aus-tausch-ver-fah-ren}
\hyphenation{Schlüs-sel-aus-tausch-ver-fah-rens}



\TUBAFFakultaet{Fakultät für Mathematik und Informatik}
\TUBAFInstitut{Institut für Informatik}
\TUBAFLehrstuhl{Lehrstuhl für Betriebssysteme und Kommunikationstechnologien}

%\TUBAFZweitlogo{\includegraphics{thekla_logo.jpg}}

\TUBAFTitel[Entwicklung einer Sicherheitsinfrastruktur zur Bluetooth-Kommunikation zwischen Smartphone und Mikrocontroller]{Entwicklung einer Sicherheitsinfrastruktur zur Bluetooth-Kommunikation zwischen Smartphone und Mikrocontroller}
\TUBAFBetreuer{\,Prof.\,Dr. Konrad Froitzheim}
\TUBAFKorrektor{M.Sc. Jonas Treumer}%TODO Betreuer Lorenzo Neumann
\TUBAFAutor[M. Käsemodel]{Marian Käsemodel}
\TUBAFStudiengang{Angewandte Informatik}
\TUBAFVertiefung{Technik}
\TUBAFMatrikel{62\,412}
\TUBAFDatum[]{\today}%TODO

\begin{document}

\maketitle

\TUBAFErklaerungsseite


\KOMAoptions{
	listof=totoc	% Abbildungs- und Tabellenverzeichnis im Inhaltsverzeichnis
}

\tableofcontents
\newpage
\listoffigures
\listoftables

\newpage
\section{Einleitung}
\label{sec: einleitung}

	\subsection{Themenstellung}
		\label{sec: themenstellung}
		\input{sections/einleitung/themenstellung.tex}

	\subsection{Problemstellung}
		\label{sec: problemstellung}
		Ziel der Arbeit ist es zunächst die Sicherheitslücken der Architektur von \textit{Bluetooth Low Energy} darzulegen.

\newpage
\section{Grundlagen zu Bluetooth Low Energy}
\label{sec: grundlagen le}

	\subsection{Überblick}
		\label{sec: le ueberblick}
		Bluetooth ist ein Industriestandard für die Übertragung von Daten per Funk, dessen Intention die Reduzierung von Kabelverbindungen an mobilen sowie stationären Geräten ist. Wichtige Eigenschaften der Technologie sind vor allem ein niedriger Energieverbrauch und günstig herstellbare Hardware. Seit 1998 wird Bluetooth von der \textit{Bluetooth Special Interest Group} (SIG) entwickelt und ist seit 2002 von der \textit{Organisation Institute of Electrical and Electronics Engineers} (IEEE) standardisiert \cite{IEEE}.
% QUELLE https://standards.ieee.org/standard/802_15_1-2002.html)
\\\\
Es agiert im lizenzfreien ISM-Band (Industrial, Scientific and Medical Band) von 2,4 GHz. 
% TODOOPT QUELLE
% BR/EDR: BT Specification 4.0 PDF S. 124, 1.1 Overview of BR/EDR Operation, 1. Absatz
% LE: BT Specifiaction 4.0 PDF S. 126, 1.2 OVERVIEW OF BLUETOOTH LOW ENERGY OPERATION, 1. Absatz
Zur Reichweite kann keine genaue Aussage getroffen werden, da sie von vielen Parametern wie beispielsweise der Sendeleistung und Einflüssen aus der Umwelt abhängt. Um trotzdem einen Eindruck zu gewinnen, kann für bestimmte Bedingungen und Konfigurationen die maximale Reichweite mithilfe eines Tools \cite{BtRangeTool} der SIG ermittelt werden. Dabei variieren die Ergebnisse von ca. einem Meter bis hin zu mehr als 1000 Metern.
% QUELLE https://www.bluetooth.com/learn-about-bluetooth/key-attributes/range/
\\\\
Grundlegend wird Bluetooth seit der Version 4.0 von 2010 in die zwei Systeme \textit{Basic Rate} (BR) und \textit{Low Energy} (LE) unterteilt, wobei LE darauf ausgelegt ist, weniger Energie als BR zu benötigen. Die neueste Bluetooth"=Version ist die Version 5.2, die wie jede Version abwärtskompatibel ist. Jedoch sind beide Systeme (BR und LE) bezüglich der Kommunikation miteinander inkompatibel: implementiert ein Gerät nur das BR"=System, kann es keine Daten mit einem Gerät austauschen, das nur das LE"=System unterstützt. Demnach ist für LE die Abwärtskompatibilität nur bis zur Version 4.0 gegeben. Desweiteren ist es möglich, dass ein Gerät über beide Systeme verfügt und so die meisten Nutzungsfälle abdeckt. Das BR"=System kann mit den Erweiterungen \textit{Enhanced Data Rate} (EDR) und \textit{Alternate Media Access Control and Physical Layer} (AMP) genutzt werden, um eine höhere Datenrate zu erzielen. Die einzelnen Systeme und Erweiterungen können entsprechend ihrer Bluetooth-Version die in Tabelle \ref{tab: maximale Bitraten BT} dargestellten Datenraten erreichen.
\begin{table}[H]
    \begin{tabular}[h]{|l|l|l|}
    \hline
    \textbf{System/Erweiterung} & \textbf{max. Bitrate (Version 4.0)} & \textbf{max. Bitrate (Version 5.2)} \\
    \hline
    BR          & 1 Mbit/s \cite{BtSpec4.0_124}               & 1 Mbit/s \cite{BtSpec5.2_188}           \\
    \hline
    BR/EDR      & 2 Mbit/s bis 3 Mb/s \cite{BtSpec4.0_124}    & 2 Mbit/s bis 3Mb/s \cite{BtSpec5.2_188} \\
    \hline
    802.11 AMP  & 24 Mbit/s \cite{BtSpec4.0_123}              & 52 Mbit/s \cite{BtSpec5.2_187}          \\
    \hline
    LE          & 1 Mbit/s \cite{BtSpec4.0_126}               & 2 Mbit/s \cite{BtSpec5.2_190}           \\
    \hline
    % QUELLE
    % BT Specification 4.0 PDF S. 124, 1.1 Overview of BR/EDR Operation, 1. Absatz
    %                          S. 126, 1.2 OVERVIEW OF BLUETOOTH LOW ENERGY OPERATION, 1. Absatz
    % BT Specification 5.2 PDF S. 188 1.1 Overview of BR/EDR Operation, 1. Absatz
    %                          S. 190, 1.2 OVERVIEW OF BLUETOOTH LOW ENERGY OPERATION, 1. Absatz
    \end{tabular}
    \caption[Maximale Bitraten der Bluetooth-Systeme]{Maximale Bitraten der Bluetooth-Systeme}
    \label{tab: maximale Bitraten BT}
\end{table}
\textit{Da Bluetooth Low Energy (BLE) ein zentraler Bestandteil dieser Arbeit ist, bezieht sich der Autor von nun an nur darauf und nicht mehr auf Bluetooth im Allgemeinen. D.h., dass Bluetooth Classic, welches BR/EDR und AMP beschreibt, nur noch behandelt wird, wenn das BR"=System oder eine seiner Erweiterungen explizit erwähnt werden.}
\\\\
Die Architektur eines Bluetooth"=Systems unterteilt sich in einen Host und in einen oder mehrere Controller. Ein Host ist eine logische Entität, definiert als alle Schichten unterhalb der nicht zu Bluetooth gehörigen Profile (Protokolle) und oberhalb des Host"=Controller"=Interface (HCI). Ein Controller ist eine logische Entität, definiert als alle Schichten bzw. Funktionsblöcke unterhalb des HCI. Der Aufbau setzt sich immer aus genau einem primären Controller und optional aus sekundären Controllern zusammen. Dabei kann die Rolle des primären Controllers entweder durch einen BR/EDR"=Controller, einen LE"=Controller oder durch eine Kombination aus BR/EDR- und LE"=Controller eingenommen werden, während die Rolle eines sekundären Controllers nur durch einen AMP"=Controller besetzt werden kann. In Abb. \ref{fig: kombinationen aus host und controller} sind einige Varianten skizziert. 
% TODOOPT QUELLE BT Specification 4.0, PDF S.123 f.
\begin{figure}[H]
    \centering
    \includegraphics[width=0.9\linewidth]{graphics/kombination_host_controller.pdf}
    \caption[Kombinationen aus Host und Controller]{Kombinationen aus Host und Controller; in Anlehnung an \cite{BtSpec4.0_fig_124}}
    \label{fig: kombinationen aus host und controller}
\end{figure}
% QUELLE BILD BT Sepcification 4.0, PDF S. 124
Zur Veranschaulichung der Architektur bezüglich eines LE"=Systems ist in der Abb. \ref{fig: host controller architektur} die Zusammensetzung aus Host und Controller mit deren Schichten bzw. Protokollen festgehalten, die in den Sektionen \ref{sec: le controller} und \ref{sec: le host} thematisiert werden. Über dem Host befindet sich die Anwendungsschicht.
\begin{figure}[H]
    \centering
    \includegraphics[width=0.7\textwidth]{graphics/host_controller_hci.pdf}
    \caption[BLE-Architektur von Host und Controller]{BLE-Architektur von Host und Controller; in Anlehnung an \cite{BtSpec4.0_fig_137}}
    \label{fig: host controller architektur}
\end{figure}

	\subsection{Topologie}
		\label{sec: le topologie}
		Die Topologie entspricht der vorgestellten Topologie der Infrastruktur (siehe Sektion \ref{sec: infra topologie}). Zusätzlich wird ein Back End in Form eines Servers benötigt, um den autonomen Verleih zu steuern. Die Abb. \ref{fig: impl topo} zeigt die Topologie der Implementierung.
\begin{figure}[H]
    \centering
    \includegraphics[width=1\textwidth]{graphics/impl_topologie.pdf}
    \caption[Topologie der Implementierung]{Topologie der Implementierung}
    \label{fig: impl topo}
\end{figure}
Bevor ein Ausleihprozess stattfinden kann, müssen Back End, Mikrocontroller und Smartphone jeweils über ein von der Zertifizierungsstelle ausgestelltes Zertifikat verfügen. Jeder Partei außer der Zertifizierungsstelle sollte periodisch (z.B. jährlich) ein Zertifikat ausgestellt werden.

	\subsection{Verbindungsaufbau}
		\label{sec: le verbindungsaufbau}
		Um mittels BLE ein Piconet zu bilden, benötigt es einen Advertiser. Auf drei vorgegebenen Frequenzen, den Advertising Channels (siehe X)
% TODO SEKTION VERWEIS grundlagen ble controller physical channel
, sendet dieser Daten, mit denen er sich für andere Geräte bemerkbar macht (Advertisements). Dabei können Advertisements auch genutzt werden, um Nutzdaten zu senden. Jedes Advertisement-Paket beinhaltet eine Bluetooth-Adresse des Senders, welche 48 Bit lang ist.

Geräte, die Daten auf den Advertising Channels empfangen, werden Scanner bzw. Initiator genannt. Auf diesem Weg finden sich die Geräte (Discovering). Der Initiator unterscheidet sich vom Scanner, da er in der Lage ist, sich zu einem Advertiser zu verbinden, von dem er ein Advertisement erhielt, dass das Verbinden zu diesem ermöglicht. Sind zwei Geräte verbunden senden und empfangen sie ihre Pakete auf den Data Channels (siehe X).
% TODO SEKTION VERWEIS grundlagen ble controller physical channel
Verbinden sich zwei Geräte wird der Initiator zum Master und der Advertiser zum Slave.
\\\\
Durch die Anwendung von Zeitmultiplexing senden die Geräte ihre Pakete immer zu festgelegten Zeitpunkten. Dabei ist ein Event ein zeitlicher Abschnitt, in dem zusammenhängende Daten in Form von Paketen gesendet bzw. empfangen werden.
% TODO BILD lies folgenden Satz, siehe auch Spec S. 127 oben
In Abbildung X ist ein Advertising Event dargestellt bei dem ein Advertiser auf allen drei Advertising Channels nacheinander Advertisement-Pakete sendet. Auf dem zweiten Kanal empfängt der Advertiser direkt gefolgt auf sein erstes Advertisement-Paket in diesem Kanal ein Paket eines Scanners, auf welches er mit einem weiteren Advertisement antwortet.
% TODO BILD Spec S. 127 unteres bild
In Abbildung X ist ein Advertising Event dargestellt, bei dem ein Initiator auf das Advertisement-Paket eines Advertiser antwortet, um eine Verbindung aufzubauen. Darauf folgt ein Connection Event bei dem Master (ursprünglich Initiator) und Slave (ursprünglich Advertiser) auf einem Data Channel sich gegenseitig Pakete senden. Danach folgt ein weiteres Connection Event auf einem anderen Data Channel.

	\subsection{Controller}
		\label{sec: le controller}
		Wie in Abbildung X
% TODO BILD VERWEIS letztes Bild aus überblick
zu sehen ist, umfasst der Controller die Schichten Physical Layer (PHY) und Link Layer (LL, auch Logical Layer genannt). Der Phyical Layer unterteilt sich weiter in die Physical Channels und die Phyical Links, während der Link Layer sich in Logical Transports und Logical Links aufteilt. In den Schichten bzw. ihren Untergliederungen wird definiert, wie Anwenderdaten, Advertisements und Kontrollsignale in Form von Unicasts bzw. Broadcasts übertragen werden.

% TODO ABBILDUNG physical layer, physical channel/transports, link layer, locigal transports/links
% TODO evtl zum Bild anmerken, dass L2CAP nicht zum Controller sondern zum Host gehört

		\subsubsection{Physical Layer}
			\label{sec: le phy}

			\paragraph{Physical Channel} \mbox{} \vspace{0.2cm} \\
				\label{sec: le phy channel}
				Um miteinander zu kommunizieren, müssen zwei Bluetooth"=Geräte (ein Sender und ein Empfänger) zur selben Zeit den selben Kanal nutzen, wobei sich der Empfänger in der Reichweite des Senders befinden muss. Da mehrere Piconets zur selben Zeit im selben lokalen Bereich agieren können, besteht die Wahrscheinlichkeit, dass zwei Sender zweier verschiedener Gerätepaare in Reichweite den selben Kanal zur selben Zeit nutzen und eine Kollision verursachen.
\\\\
Mittles des Frequenzmultiplexverfahrens ist das ISM"=Band eines LE"=Systems von 2400 MHz bis 2483,5 MHz in 40 Funkkanäle unterteilt. Beginnend bei 2402 MHz nutzt jeder Kanal eine Frequenz, die 2 MHz über der Frequenz des Vorgängers liegt. Das Trägersignal wird mithilfe des Gaussian Frequency Shift Keying moduliert. \cite{BtSpec4.0_2180-2181}
% QUELLE Specification 4.0 PDF S. 2180-2181

Somit bildet der Physical Channel die niedrigste Ebene der Architektur. 37 der 40 Kanäle werden als LE Piconet Channel (entsprechend S. \pageref{fig: controller architektur} Abb. \ref{fig: controller architektur} als LE Piconet Physical Channel) bezeichnet, die mit einem Piconet assoziiert werden und zur Kommunikation zwischen zwei bereits verbundenen Geräten dienen. Die verbleibenden drei Kanäle werden Advertisement Broadcast Channel (entsprechend S. \pageref{fig: controller architektur} Abb. \ref{fig: controller architektur} als LE Advertising Physical Channel) genannt und befinden sich auf den Frequenzen 2402 MHz, 2426 MHz sowie 2480 MHz. \cite{BtSpec4.0_2199}
% QUELLE advertisment frequenzen Specification 4.0 PDF S. 2199
\\\\
Mittels Advertisements können Geräte in diesen drei Kanälen auf sich aufmerksam machen, um von anderen Geräten entdeckt zu werden. Zudem werden sie genutzt, um Geräte miteinander zu verbinden oder Anwendungsdaten an Scanner bzw. Initiatoren zu senden. Ein Gerät kann nur einen Kanal zur selben Zeit nutzen, weswegen das Zeitmultiplexverfahren verwendet wird, das bereits verbundenen Geräten ermöglicht, zusätzlich das Advertisement zu betreiben.
\\\\
Um Interferenzen innerhalb des genutzten Frequenzbands z.B. mit \textit{Wi-Fi} zu vermeiden, wird das Adaptive Frequency Hopping \cite{BtAfh} genutzt (eine Form des Frequency Hopping Spread Spectrum). Dabei wechseln Sender und Empfänger in kurzen Zeitabständen den Kanal und passen die Menge der zu nutzenden Kanäle (Channel Map) an, indem sie dynmaisch ermittleln, in welchen Kanälen häufiger Kollisionen auftreten. Treten in einem Kanal häufig Interferenzen auf, wird er für eine bestimmte Zeitspanne aus der Channel Map entfernt und vorerst nicht mehr genutzt.
% QUELLE https://www.bluetooth.com/blog/how-bluetooth-technology-uses-adaptive-frequency-hopping-to-overcome-packet-interference/

			\paragraph{Physical Link} \mbox{} \vspace{0.2cm} \\
				\label{sec: le phy link}
				Ein Physical Link wird immer mit genau einem Physical Channel assoziiert. Dagegen kann ein Physical Channel mehrere Physical Links unterstützen. Bezüglich Bluetooth wird der Physical Link nicht in der Struktur eines Paketes repräsentiert, kann aber innerhalb eines LE-Paketes anhand der Access Address identifiziert werden. \cite{BtSpec4.0_164}
\\\\
Active Physical Links sind die Punkt-zu-Punkt-Verbindungen zwischen Master und Slave über einen Piconet Physical Channel und gelten nur als aktiv, wenn ein Asynchronous Connection (ACL) Logical Transport zwischen den Geräten existiert. \cite{BtSpec4.0_166-167}

Advertising Physical Links dienen dazu, zwischen einem Advertiser und einem Initiator einen Active Physical Link aufzubauen, und existieren nur für einen kurzen Zeitraum. Zwischen Advertiser und Scanner existieren sie für längere Zeit und dienen dem Broadcast von Nutzdaten. \cite{BtSpec4.0_166-167}

		\subsubsection{Link Layer}
			\label{sec: le ll}

			\paragraph{Paketstruktur}
				\label{sec: le ll paketstruktur}
				Der Link Layer nutzt ein gemeinsames Paketformat für das Übertragen von Advertising-Paketen und Nutzdaten-Pakten, das in Abb. \ref{fig: ll paket struktur} dargestellt ist.\\

\begin{figure}[H]
    \centering
    \includegraphics[width=0.9\textwidth]{graphics/link_layer_packetformat.pdf}
    \caption[Paketstruktur des Link Layers]{Paketstruktur des Link Layers; in Anlehnung an \cite{BtSpec4.0_fig_2200}}
    \label{fig: ll paket struktur}
\end{figure}
% QUELLE Spec 4.0 S. 2200 - 2201

Die Preamble hat eine Größe von acht Bit und wird genutzt, um auf Empfängerseite die Frequenz zu synchronisieren, die Zeiteinteilung der Symbole zu schätzen und um die Automatic Gain Control zu trainieren. Die Preamble beträgt immer 0b01010101, falls das Bit mit dem niedrigsten Stellenwert (LSB für Least Significant Bit) der Access Address 1 ist. Anderenfalls beträgt die Preamble 0b10101010. \cite{BtSpec4.0_2200-2201}
\\\\
Die Access Address hat eine Größe von 32 Bit und identifziert eine Verbindung über den Link Layer bzw. dient dazu Pakete mittels des festgelegten Wertes 0x8E89BED6 als Advertisement-Pakete zu identifzieren. Bevor ein Initiator eine Verbindung zu einem Advertiser aufbaut, erstellt er eine zufällige Access Address, die neben weiteren Bedingungen nicht der des Advertisement"=Pakets gleicht oder sich von dieser um ein Bit unterscheidet. Diese Access Address sendet er dann innerhalb der Verbindungsanfrage an den Advertiser. \cite{BtSpec4.0_2200-2201}
\\\\
Das letzte Feld des Link-Layer-Pakets ist der 24 Bit lange Cyclic Redundancy Check (CRC), der über das PDU-Feld berechnet wird. Im Fall, dass auf Ebene des Link Layer die PDU verschlüsselt wird, wird der CRC erst nach der Verschlüsselung generiert. Unabhängig davon, ob verschlüsselt wird oder nicht, wird anschließend ein Whitening \cite{BtSpec4.0_2217-2218} durchgeführt, um Sequenzen vieler gleichbleibender Bits (bspw. 0b00000000) zu verhindern. \cite{BtSpec4.0_2200-2201}
% QUELLE Spec 4.0 S. 2217-2218
\\\\
Die Protocol Data Unit (PDU) unterscheidet sich in die Advertising Channel PDU und Data Channel PDU.

\subparagraph{Advertising Channel PDU} \mbox{} \vspace{0.2cm} \\
Wie in Abb. \ref{fig: ll adv channel pdu} gezeigt wird, besteht die Advertising Channel PDU aus einem 16 Bit langen Header und einem Payload variabler Länge. RFU steht dabei für Reserved for Future Use.

\begin{figure}[H]
    \centering
    \includegraphics[width=0.9\textwidth]{graphics/link_layer_packetformat_pdu_adv.pdf}
    \caption[Link Layer Advertising Channel PDU]{Link Layer Advertising Channel PDU; in Anlehnung an \cite{BtSpec_fig_2201} und \cite{BtSpec_fig_2202}}
    \label{fig: ll adv channel pdu}
\end{figure}
% QUELLE advertising channel pdu format S. 2201 f.

Dabei beinhaltet der Header unter anderem ein 4 Bit langes Feld für den PDU Type (bspw. Connectable Undirected Advertising Event oder Scan Request) und zwei Flags TxAdd und RxAdd für zusätzliche Informationen bezüglich des PDU Type. Die Bedeutungen von TxAdd und RxAdd hängen vom PDU Type ab. Die Menge aller PDU Types lässt sich untergliedern in Advertising PDUs, Scanning PDUs und Initiating PDUs. Bei allen bilden die ersten 6 Bytes des Payload die Adresse des Senders (Advertiser, Scanner oder Initiator). Hier sagt TxAdd bei jedem PDU Type aus, ob die angegebene Adresse des Senders öffentlich (TxAdd = 0) oder zufällig generiert (TxAdd = 1) ist. Diese Funktion wird Privacy Feature gennant und ist in Sektion \ref{sec: gap sicherheit} genauer beschrieben. RxAdd dagegen ist nur bei PDU Types von Bedeutung, die in ihrem Payload eine zweite Adresse enthalten, nämlich die des Empfängers. Analog zu TxAdd sagt RxAdd aus, ob die Adresse des Empfängers öffentlich (RxAdd = 0) oder zufällig (RxAdd = 1) ist.

Ein weiteres Feld im Header der Advertising Channel PDU ist das 6 Bit lange Feld für die Länge des Payloads in Bytes, dessen Wert eine Spanne von 6 bis 37 Byte deckt. \cite{BtSpec4.0_2201-2208}
\\\\
\subparagraph{Data Channel PDU} \mbox{} \vspace{0.2cm} \\
Die Data Channel PDU nutzt entsprechend der Abb. \ref{fig: ll data channel pdu} einen 16 Bit langen Header, einen Payload variabler Länge und optional einen 32 Bit langen Message Integry Check (MIC), der die Integrität des Payload sicherstellt. Das MIC-Feld entfällt bei einer unverschlüsselten Link-Layer-Verbindung und bei einer Data Channel PDU, deren Payload die Länge null beträgt.

\begin{figure}[H]
    \centering
    \includegraphics[width=1\textwidth]{graphics/link_layer_packetformat_pdu_data.pdf}
    \caption[Link Layer Data Channel PDU]{Link Layer Data Channel PDU; in Anlehnung an \cite{BtSpec_fig_2208a} und \cite{BtSpec_fig_2208b}}
    \label{fig: ll data channel pdu}
\end{figure}

Das erste Feld des Header ist der 2 Bit lange Link Layer Identifier (LLID), der mit 0b01 sowie 0b10 aussagt, dass es sich um eine LL Data PDU handelt und mit 0b11, dass es sich um eine LL Control PDU handelt. Der Wert 0b00 ist reserviert. Die LL Control PDU dient dazu, um die LL-Verbindung zu steuern. Dazu gehören unter anderem Anfragen zum Ändern der Verbindungsparameter (z.B. Window Size oder Wert bis zur Zeitüberschreitung), zum Ändern der Channel Map oder zum Verschlüsseln.

Auf die LLID folgt das Feld der Next Expected Sequence Number (NESN) und das Feld der Sequence Number (SN) mit jeweils einem Bit Länge, die innerhalb Sektion \ref{sec: le ll transport} näher erläutert werden.

Unter anderem beinhaltet der Header ein 5 Bit langes Feld für die Länge des Payload in Byte und ggf. einschließlich der Länge des MIC. Der maximale Wert der Länge beträgt 31 Byte, wobei sich der Payload in jedem Fall auf eine maximale Länge von 27 Byte bemisst. \cite{BtSpec4.0_2208-2209}

In Bluetooth Version 4.2 ist das Feld der Länge auf 8 Bit erweitert, wodurch der Payload eine maximale Größe von 251 Byte annehmen kann. \cite{BtSpec4.2_2589-2590}

			\paragraph{Logical Transport} \mbox{} \vspace{0.2cm} \\
				\label{sec: le ll transport}
				Über dem Physical Layer baut sich der Link Layer auf, beginnend mit dem Logical Transport, der sich in die zwei Arten LE Asynchronous Connection (LE ACL) und LE Advertising Broadcast (ADVB) unterteilt.

Die LE ACL transportiert Kontrollsignale des über ihr befindlichen Logical Link und Logical Link Control and Adaption Protocol (L2CAP). Außerdem überträgt die LE ACL asynchrone Anwenderdaten nach dem Best-Effort-Prinzip.

Mithilfe der Next Expected Sequence Number bzw. Sequence Number (NESN/SN), die jeweils nur die Größe eines Bits besitzen, wird eine einfache Zuverlässigkeit gewährleistet. Empfängt ein Gerät ein Paket B, vergleicht es dessen NESN mit der SN, die es innerhalb des vorherigen Pakets A abgesendet hat. Wenn diese unterschiedlich sind, wurde das vorherige Paket A vom Gegenüber vollständig und korrekt empfangen (ACK für Acknowledgement). Anderenfalls sind die Nummern gleich, was bedeutet, dass das vorherige Paket A nicht vollständig bzw. korrekt empfangen wurde (NACK/NAK für Negative Acknowledgement) und nun erneut an den Gegenüber gesendet werden muss. Zudem prüft das Gerät die SN des empfangenen Pakets B mit der NESN seines vorher gesendeten Pakets A. Sind diese Nummern gleich, wurde das empfangene Paket B vom Gerät erwartet. Anderenfalls sind die Nummern verschieden und somit wurde das Paket B nicht erwartet, weswegen es ignoriert wird. Somit wird auch die Flusskontrolle ermöglicht, da ein Empfänger bei nicht ausreichend freiem Speicher im Buffer ein NACK mithilfe der Sequenznummern zurücksenden kann.
% TODO QUELLE evtl mit BILD: NESN/SN 4.0 PDF S. 2240 f.

Wenn ein Gerät einem Piconet beitritt, wird zwischem dem Master und dem Slave eine Default LE ACL über einen Active Physical Link gebildet. Die Default LE ACL ist einer Access Address zugeordnet. Wird die Default LE ACL getrennt, werden alle Logical Transports zwischen Master und Slave getrennt. Bei einem unerwarteten Synchronisationsverlust zum LE Piconet Physical Channel werden der LE Physical Link und alle LE Logical Transports und LE Logical Links entfernt.

Der ADVB transportiert ohne Verwendung von Acknowledgements Kontrollsignale und Anwenderdaten bezüglich des Broadcasts über den darunter gelegenen LE Advertising Broadcast Link. Der Datenverkehr ist überwiegend unidirektional, ausgehend vom Advertiser zu allen in Reichweite befindlichen Scannern. Scanner können eine Anfrage an den Advertiser senden, um weitere Anwenderdaten über den Broadcast zu empfangen oder um eine LE ACL zu bilden. Aufgrund des Verzichts auf Acknowledgements ist der ADVB unzuverlässig, weswegen Pakete redundant übertragen werden. Sobald ein Gerät mit dem Advertising beginnt, wird ein ADVB erzeugt, der anhand der Adresse des Gerätes identifiziert wird.
% TODO QUELLE allg. für paragraph: Specification 4.0 PDF S. 174

			\paragraph{Logical Link} \mbox{} \vspace{0.2cm} \\
				\label{sec: le ll link}
				Ein Logical Link unterscheidet sich je nachdem, ob er auf einem LE ACL Logical Transport oder einem ADVB Logical Transport aufbaut und ob er zur Übertragung von Kontrollsignalen oder Nutzdaten genutzt wird. Anhand des Logical Link Identifier (LLID) im Header des Basisbandpakets wird unterschieden, ob es sich bei der zu übertragenden PDU um Nutzdaten oder Kontrollsignale handelt.
% TODO Bildverweis einfügen Paketstruktur

Der Control Logical Link (LE-C) nutzt den darunter liegenden LE ACL Logical Transport, um Kontrollsignale zwischen Geräten im Piconet zu übertragen.

Der User Asynchronous Logical Link (LE-U) nutzt den darunter liegenden LE ACL Logical Transport, um alle asynchronen Anwenderdaten zu übertragen. Über dem Link Layer agiert das Protokoll L2CAP, dessen Frame für den Link Layer fragmentiert werden müssen. Mithilfe des LLID-Wertes 0b10 wird der Beginn eines L2CAP-Frame (das erste Fragment eines L2CAP-Frame) und der Wert 0b01 die Fortsetzung eines L2CAP-Frame (die folgenden Fragmente des L2CAP-Frame) gekennzeichnet. Somit wird der Header des Protokolls L2CAP einfach gehalten und eine korrekte Synchronisation bei der Zusammensetzung der Fragmente zu einem L2CAP-Frame garantiert. Jedoch muss folglich ein L2CAP-Frame vollständig übertragen werden, bevor ein neues übertragen wird.

Der Advertising Broadcast Control Logical Link (ADVB-C) nutzt den darunter liegenden Default ADVB, um Kontrollsignale für Verbindungsanfragen oder Anfragen für weitere Broadcast-Anwenderdaten zu übertragen.

Der Advertising Broadcast User Data Logical Link (ADVB-U) nutzt den darunter liegenden Default ADVB, um verbindungslos und ohne den Gebrauch von LE-U Anwenderdaten als Broadcast zu senden.
%TODO QUELLE paragraph: Specification 4.0, S.176 f.

	\subsection{Host}
		\label{sec: le host}
		Im Wesentlichen umfasst der Host das Logical Link Control and Adaption Protocol (L2CAP), das Generic Access Profile (GAP) und Generic Attribute Profile (GATT) sowie das Security Manager Protocol (SMP). Er liegt unter der Anwendungsebene und steuert den Datentransport als auch den Verbindungsaufbau sowie Sicherheitsaspekte. Je nachdem wie Host und Controller implementiert sind, kann das standarisierte Host Controller Interface (HCI) als Schnittstelle zwischen Host und Controller dienen. Es ist optional \cite{BtSpec4.0_138}, da es Implementierungen bestehend aus Host und Controller gibt, die direkt verbunden sind, so dass ein HCI nicht nötig ist. Da das HCI den allgemeinen Kommunikationsprozess zwischen Bluetooth-Geräten nicht nennenswert beeinflusst, wird es hier nicht weiter behandelt.

		\subsubsection{Logical Link Control and Adaption Protocol}
			\label{sec: le l2cap}
			Das Logical Link Control and Adaption Protocol (L2CAP) bildet die unterste Schicht im Host (siehe S. \pageref{fig: host controller architektur} Abb. \ref{fig: host controller architektur}) und dient je nach Konfiguration dazu, den Datenverkehr zu steuern und zwischen höheren und niedrigeren Schichten zu vermitteln. Mittels Multiplexing können mehrere Anwendungen einen LE-U (also Logical Link) nutzen.
% TODOOPT QUELLE Spec 4.0 S. 1400 2.3 Operation Between Layers
Es verfügt über fünf Modi:
\begin{itemize}
    \item Basic L2CAP Mode
    \item Flow Control Mode
    \item Retransmission Mode
    \item Enhanced Retransmission Mode
    \item Streaming Mode
    \item LE Credit Based Flow Control Mode (seit Bluetooth 4.2)
\end{itemize}
Für Bluetooth allgemein (BR/EDR und LE) wird immer der Basic L2CAP Mode genutzt, wenn kein anderer festgelegt wird. Der LE Credit Based Flow Control Mode soll \cite{BtSpec4.2_1735} zufolge als einziger Modus für verbindungsorientierte LE"=Kanäle genutzt werden ("`This is the only mode that shall be used for LE L2CAP connection oriented channels"' \cite{BtSpec4.2_1735}). Da diese Aussage nicht ausschließt, dass eine verbindungsorientierte LE"=Verbindung über den standardmäßig festgelegten Basic L2CAP Mode erfolgen kann, und der LE Credit Based Flow Control Mode noch nicht in der Bluetooth"=Version 4.0 vertreten war \cite{BtSpec4.0_1401}, ist anzunehmen, dass eine verbindungsorientierte LE"=Verbindung auch mit dem Basic L2CAP Mode möglich ist.
% QUELLE ZITAT Spec. 4.2 S. 1735
% QUELLE Spec 4.0 S. 1401
\\\\
Logische Kanäle, genannt L2CAP Channels, dienen innerhalb eines Geräts als Endpunkt für höher gelegene Protokolle oder direkt für die Anwendung und sind für jedes Gerät individuell an dem Channel Identifier (CID) unterscheidbar. Folglich muss der L2CAP Channel einer Verbindung zwischen zwei Geräten von diesen nicht zwingend mit der gleichen CID gekennzeichnet sein.
% TODOOPT QUELLE Spec. 4.0 S. 1390
\\\\
Der L2CAP Layer wird von zwei Modulen gesteuert: dem Resource Manager und dem Channel Manager (siehe Abb. \ref{fig: l2cap architektur}).

\begin{figure}[H]
    \centering
    \includegraphics[width=0.9\textwidth]{graphics/l2cap_architektur.pdf}
    \caption[Architektur des L2CAP Layers]{Architektur des L2CAP Layers \cite{BtSpec4.0_1391}}
    \label{fig: l2cap architektur}
\end{figure}
% QUELLE l2cap layer resource manager channel manager spec 4.0 S. 1391

Der Channel Manager ist für die Verwaltung und Steuerung der L2CAP Channels zuständig. Das umfasst die auf L2CAP bezogene Signalübertragung intern, Peer"=to"=Peer, und zu höheren und niedrigeren Schichten \cite{BtSpec4.0_1390}. 
% QUELLE Spec. 4.0 S. 1390
Die Signale, die zwischen den zwei L2CAP"=Entitäten zweier verbundenen Geräte übertragen werden, sind Kommandos wie bspw. der LE Credit Based Connection Request und die entsprechende Response. Dafür wird ein separater L2CAP Channel mit der CID 0x0005 genutzt. Zudem betreibt der Channel Manager den L2CAP"=Zustandsautomaten, auf den hier nicht näher eingegangen werden soll.
\\\\
Da der L2CAP Layer nach unten mit dem Link Layer verknüpft ist (ggf. über das HCI), müssen die L2CAP PDUs dem Paketformat des Link Layers (bzw. dem HCI) gerecht werden. Dementsprechend werden die L2CAP PDUs fragmentiert bzw. wieder zusammengesetzt. Die Maximal PDU Payload Size (MPS) bezeichnet die maximale Größe des Payload einer PDU in Byte, die eine L2CAP"=Entität verarbeiten kann.
% TODOOPT BILD + VERWEIS Spec 4.0 S. 1487, Mischung aus Figure 7.1 und 7.2, vereinfacht darstellen: L2CAP PDU -> HCI Paket -> Link Layer Paket
\\\\
Ein Datenaustausch zwischen L2CAP und dessen höhergelegenen Schichten / Protokollen erfolgt in der Form von Service Data Units (SDU), wobei deren Ursprung immer aus einer höheren Schicht stammt. Die Maximum Transmission Unit (MTU) bezeichnet die maximale Größe einer SDU in Byte, die die höhergelegene Schicht verarbeiten kann. Die Segmentierung (bzw. Zusammensetzung) dieser SDUs wird vom Resource Manager ausgeführt und ist für alle Modi außer dem Basic L2CAP Mode relevant. Wenn keine Segementierung angewandt wird, ist die MTU gleich der MPS \cite{BtSpec4.2_1727}. Falls ein L2CAP Channel in einem anderen Modus als dem Basic L2CAP Modus agiert, ist der Resource Manager auch für die erneute Übertragung von PDUs und Flusskontrolle zuständig. Außerdem plant der Resource Manager zu welchem Zeitpunkt L2CAP Channel PDUs versenden können, um den L2CAP Channels mit Quality"=of"=Service"=Optionen genügend Ressourcen bezüglich der Buffer des Controllers freizugeben. \cite{BtSpec4.2_185} \cite{BtSpec4.2_1725-1726}

\paragraph{Struktur eines Datenpakets} \mbox{} \vspace{0.2cm} \\
Es existieren verschiedene Arten von L2CAP"=Datenpaketen, die den gennanten Modi zugeordnet sind. Der Basic L2CAP Mode unterstützt verbindungsorientierte und verbindungslose L2CAP Channel, während die restlichen Modi nur verbindungsorientierte L2CAP Channels nutzen.
In Abb. \ref{fig: l2cap pdu basic} ist die Paketstruktur einer L2CAP PDU im Basic L2CAP Mode (verbindungsorientiert) dargestellt.

\begin{figure}[H]
    \centering
    \includegraphics[width=0.9\textwidth]{graphics/l2cap_datenpaket.pdf}
    \caption[Struktur einer L2CAP PDU (Basic L2CAP Mode)]{Struktur einer L2CAP PDU (Basic L2CAP Mode) \cite{BtSpec4.2_fig_1737}}
    \label{fig: l2cap pdu basic}
\end{figure}
% QUELLE Spec. 4.2 S. 1737 Figure 3.1

LSB steht für Least Significant Bit, also das Bit mit niedrigstem Stellenwert, und MSB für Most Significant Bit, also das Bit mit höchstem Stellenwert. Das Längenfeld beschreibt die Länge des Payload, der eine maximale Länge von 65535 Byte besitzt.
\\\\
Abb. \ref{fig: l2cap pdu credit} zeigt die Paketstruktur einer L2CAP PDU im LE Credit Based Flow Control Mode.

\begin{figure}
    \centering
    \includegraphics[width=0.9\textwidth]{graphics/l2cap_datenpaket_credit_based.pdf}
    \caption[Struktur einer L2CAP PDU (LE Credit Base Flow Control Mode)]{Struktur einer L2CAP PDU (LE Credit Base Flow Control Mode); *nur in erster PDU einer SDU; \cite{BtSpec4.2_fig_1747}}
    \label{fig: l2cap pdu credit}
\end{figure}
% QUELLE Spec 4.2 S. 1747 Figure 3.6

Das SDU"=Längenfeld ist nur in der ersten PDU einer SDU enthalten, um die Gesamtlänge der SDU in Byte zu beschreiben. Dadurch kann die Länge des Payloads einer solchen ersten PDU nur maximal 65533 Byte betragen, da der Wert des Längenfelds des Basic L2CAP Header auch das SDU"=Längenfeld umfasst. Alle zur SDU zugehörigen folgenden PDUs enthalten das SDU"=Längenfeld nicht, wodurch deren maximale Payload"=Länge 65535 Byte beträgt.

Im LE Credit Based Flow Control Mode regulieren die beiden Endpunkte mithilfe von Credits wie viele PDUs der jeweilige Endpunkt empfangen kann. Sind keine Credits mehr für einen Endpunkt (Empfänger) vorhanden, kann der andere Endpunkt (Sender) keine PDUs mehr an den Empfänger übermitteln und muss warten bis der Empfänger ein Credit"=Signalpaket sendet, das neue Credits freigibt. \cite{BtSpec4.2_1780}

		\subsubsection{Generic Attribute Profile}
			\label{sec: le gatt}
			Das Generic Attribute Profile (GATT) dient zur Kommunikation zwischen Client und Server. Es basiert auf dem Attribute Protocol (ATT), welches ausgehend vom Server Attribute bereitstellt, die von einem oder mehreren Clients "`entdeckt"' werden können. Ein Attribut besteht aus einem Typ, der anhand des Universal Unique Identifier (UUID) identifiziert wird, einem Attribute Handle, um auf das Attribut zuzugreifen, und einem Wert, der von Server und Client ausgelesen bzw. überschrieben werden kann. Zudem ist es möglich, für Attribute Berechtigungen festzulegen (bspw. nur Lesen, nicht Schreiben). \cite{BtSpec4.0_1835}
% QUELLE Spec 4.0 S. 1835 3.1 Introduction
\\\\
Entsprechend der S. \pageref{fig: host controller architektur} Abb. \ref{fig: host controller architektur} ordnet sich ATT über L2CAP in den Protokollstapel ein.
\\\\
Möchte ein Client einen Attributwert lesen bzw. schreiben, dann sendet er einen Read Request bzw. einen Write Request an den Server. Dieser reagiert, indem er bei einem Read Request das Attribut an den Client sendet bzw. bei einem Write Request den Attributwert entsprechend ändert und dem Client eine Bestätigung sendet. Im Fall, dass ausgehend vom Server ein Attribut geändert werden soll, sendet dieser eine Notification oder eine Indication an den Client. Im Gegensatz zur Notification bestätigt der Client eine empfangene Indication. \cite{BtSpec4.0_1854-1855} \cite{BtSpec4.0_1861-1863}
% QUELLE Spec 4.0 S. 1854 f. 3.4.4.3 Read Request und 3.4.4.4 Read Response
% QUELLE Spec 4.0 S. 1861-1863 3.4.5.1 Write Request und 3.4.5.2 Write Response
\\\\
Die Daten werden in Form von Attribute Protocol PDUs (siehe Abb. \ref{fig: att pdu}) übertragen.

\begin{figure}[H]
    \centering
    \includegraphics[width=0.9\textwidth]{graphics/att_pdu.pdf}
    \caption[(Generic) Attribute Protocol PDU]{(Generic) Attribute Protocol PDU; in Anlehnung an \cite{BtSpec4.0_fig_1888} und \cite{BtSpec4.0_fig_1889}}
    \label{fig: att pdu}
\end{figure}
% Quelle Spec 4.0 S. 1888 Figure 2.3 und S. 1889 Firgure 2.4

Der ein Byte lange Opcode sagt aus, ob die PDU entweder ein Request, eine Response, Notification, Indication oder Bestätigung ist, und enthält ein Flag für die Authentifizierung. Die Attributparameter unterteilen sich in zwei Byte für das Attribute Handle, zwei oder 16 Byte für den Attribute Type (die UUID), den Attributwert variabler Länge und die Attributberechtigungen, deren Länge von der Implementierung abhängig ist. Auf die Attributparameter folgt die 12 Byte lange Signatur für die Authentifizierung, falls gefordert. \cite{BtSpec4.0_1888-1889}
% QUELLE Spec 4.0 S. 1888 f.
\\\\

\begin{figure}{H}
    \centering
    \includegraphics[width=0.9\textwidth]{graphics/gatt_hierarchie.pdf}
    \caption[]{\cite{BtSpec4.0_fig_1892}}
    \label{fig: gatt hierarchie}
\end{figure}
% Quelle Spec 4.0 S. 1892

GATT bildet eine Hierachie (siehe Abb. \ref{fig: gatt hierarchie}) bestehend aus den grundlegenden Elementen Profile, Service und Characteristic, die alle als Attribute definiert werden. An oberster Stelle befindet sich das Profile. Es enthält einen oder mehrere Services. Ein Service kann wiederum eine oder mehrere Characteristics enthalten, die sich aus Properties, einem Wert und Descriptors zusammensetzen.

		\subsubsection{Generic Access Profile}
			\label{sec: le gap}
			% TODO
Modes

Das Generic Access Profile (GAP) definiert verschiedene Rollen und Modi bzw. Prozeduren für Broadcasts und den Aufbau von Verbingungen.

Für LE existieren die vier Rollen: Broadcaster, Observer, Peripheral und Central. Ein Broadcaster ist aus Sicht des Link Layers (LL) ein Advertiser, da er verbindungslos Daten in Form von Advertising Events sendet. Der zugehörige Modus ist der Broadcast Mode. Diese Advertising Events können von Observern, die die Observation Procedure ausführen, empfangen werden, weswegen diese aus Sicht des LL als Scanner bezeichnet werden. Die Rolle Peripheral wird einem Gerät zugewiesen, wenn es den Aufbau eines LE Physical Link akzeptiert. Dabei nimmt es in Bezug auf den Link Layer die Rolle des Slave ein. Die Rolle Central wird einem Gerät zugewiesen, wenn dieses den Aufbau einer physischen Verbindung einleitet. Dabei nimmt es in Bezug auf den Link Layer die Rolle des Master ein. Ein Gerät kann mehrere Rollen zur selben Zeit einnehmen.
% TODO QUELLE Spec 4.0 S. 1638 f. 2.2.2 Roles when Operating over an LE Physical Channel
% TODO QUELLE Spec 4.0 S. 1695 f. 9.9.1 und 9.9.2

Jedes Gerät befindet sich entweder im Non-discoverable Mode, in dem es nicht von anderen Geräten entdeckt werden kann, oder im General Discoverable Mode bzw. im Limited Discoverable Mode, in denen es entdeckbar ist. Im Letzteren ist ein Gerät nur für eine bestimmte Dauer entdeckbar. Geräte, die andere Geräte entdecken sollen, müssen die General Discovery Procedure bzw. Limited Discovery Procedure ausführen.
% TODO QUELLE Spec 4.0 S. 1697 9.2 Discovery Modes And Procedures

Um Verbindungen und deren Aufbau zu steuern, gibt es mehrere Modi und Prozeduren, von denen einige in der Tabelle X zusammengefasst werden.
% TODO TABELLE VERWEIS

\begin{table}
    \begin{tabularx}{\textwidth}{|p{4.5cm}|X|}
    \hline
    \textbf{Modus/Prozedur} & \textbf{Beschreibung} \\
    \hline
    Non-connectable Mode & keine Verbindungen akzeptieren \\
    \hline
    Directed Connectable Mode & nur Verbindungen von bekannten Peer-Geräten akzeptieren, die die Auto oder General Connection Establishment Procedure ausführen \\
    \hline
    Undirected Connectable Mode & nur Verbindungen von Geräten akzeptieren, die die Auto oder General Connection Establishment Procedure ausführen \\
    \hline
    Auto Connection Establishment Procedure & Aufbau von Verbindungen zu Geräten, die in einem Connectable Mode sind und deren Adresse auf der Whitelist eingetragen ist \\
    \hline
    General Connection Establishment Procedure & Aufbau von Verbindungen zu bekannten Peer-Geräten, die in einem Connectable Mode sind \\
    \hline
    Connection Parameter Update Procedure & Peripheral oder Central kann Link-Layer-Parameter einer Verbindung ändern \\
    \hline
    \end{tabularx}
    \caption{Modi und Prozeduren für Verbindungen (Generic Access Profile)}
\end{table}
% TODO QUELLE Spec 4.0 S. 1704 - 1718 9.3 Connection Modes and Procedures

Zusätzlich verfügt GAP über Sicherheitsaspekte, die in Sektion X behandelt werden.
% TODO SEKTION VERWEIS BLE Sicherheit

		\subsubsection{Security Manager}
			\label{sec: le sm}
			Der Security Manager (SM) bzw. das Security Manager Protocol (SMP) ist dafür zuständig, eine sichere Verbindung zwischen zwei Geräten (Master und Slave) aufzubauen. Dies umfasst im Wesentlichen das Pairing, das zur Authentifizierung und Generierung eines Schlüssels dient, und die Verteilung von Schlüsseln. Entsprechend der Abb. \ref{fig: smp in bt} lässt sich der Security Manager bzw. das SMP in die BLE-Architektur einordnen.

\begin{figure}[H]
    \centering
    \includegraphics[width=0.8\textwidth]{graphics/smp_in_bt.pdf}
    \caption[Beziehungen des Security Managers zu anderen Host-Komponenten]{Beziehungen des Security Managers zu anderen Host-Komponenten; in Anlehnung an \cite{BtSpec4.0_fig_1958}}
    \label{fig: smp in bt}
\end{figure}
% QUELLE Spec 4.0 S. 1958

Um mit dem Link Layer zu interagieren, nutzt der SM einen L2CAP Channel mit einer festgelegten CID. Zudem kann er mit GAP direkt kommunizieren.
\\\\
Das Pairing lässt sich in drei Phasen aufteilen. Phase 1 und 2 dienen der Generierung eines Schlüssels, den beide Geräte zur Verschlüsselung und Authentifizierung im Link Layer nutzen. Dafür wird der Cipher Block Chaining - Message Authentication Code (CCM) Mode in Verbindung mit dem Advanced Encryption Standard (AES), genauer dem AES-128 Block Cipher, genutzt \cite{BtSpec4.0_2285}.
% QUELLE Spec. 4.0 S. 2285, 1 ENCRYPTION AND AUTHENTICATION OVERVIEW

\paragraph{Pairing: Phase 1} \mbox{} \vspace{0.2cm} \\
Die erste Phase ist der Pairing Feature Exchange, bei dem die beiden Geräte ihre für den Nutzer zugänglichen Ein- und Ausgabemöglichkeiten (IO Capabilities) austauschen. Die Tabellen \ref{tab: i caps geraet} und \ref{tab: o caps geraet} 
listen die entsprechenden Bezeichnungen dieser Möglichkeiten auf.
\\\\

\begin{table}[H]
    \begin{tabularx}{\textwidth}{|l|X|}
    \hline
    \textbf{Eingabemöglichkeit} & \textbf{Beschreibung} \\
    \hline
    Keine Eingabe & Gerät kann keine Eingaben entgegennehmen \\
    \hline
    Ja / Nein & Gerät kann zwei verschiedene Eingaben verarbeiten, die der Bedeutung Ja bzw. Nein zugewiesen werden können (bspw. zwei Tasten) \\
    \hline
    Tastatur & Eingabe der Ziffern 0 bis 9 möglich und die Möglichkeit Ja und Nein einzugeben \\
    \hline
    \end{tabularx}
    \caption[Eingabemöglichkeiten eines Gerätes]{Eingabemöglichkeiten eines Gerätes \cite{BtSpec4.0_1965}}
    \label{tab: i caps geraet}
\end{table}

\begin{table}[H]
    \begin{tabularx}{\textwidth}{|l|X|}
    \hline
    \textbf{Ausgabemöglichkeiten} & \textbf{Beschreibung} \\
    \hline
    Keine Ausgabe & Gerät kann dem Nutzer keine sechsstellige Dezimalzahl anzeigen bzw. kommunizieren \\
    \hline
    Numerische Ausgabe & Gerät kann dem Nutzer eine sechsstellige Dezimalzahl anzeigen bzw. kommunizieren \\
    \hline
    \end{tabularx}
    \caption[Ausgabemöglichkeiten eines Gerätes]{Ausgabemöglichkeiten eines Gerätes \cite{BtSpec4.0_1965_b}}
    \label{tab: o caps geraet}
\end{table}
% QUELLE für Tabellen: Spec 4.0 S. 1965 Table 2.1 und Table 2.2

Desweiteren tauschen die Geräte in der ersten Phase Informationen darüber aus, ob eine Authentifizierung zum Schutz vor einem Man-In-The-Middle-Angriff nötig ist (MITM Flag), und ob Daten für die Authentifizierung über die Pairing-Methode Out Of Band (OOB), d.h. mittels einer anderen Technologie (z.B. Near Field Communication), übertragen werden können (OOB Flag). Außerdem werden die minimalen und maximalen Größen für die Schlüssel ausgetauscht, die sich in 8 Bit großen Schritten zwischen 56 Bit (7 Byte) und 128 Bit (16 Byte) befinden. Dabei wird das Minimum beider maximaler Größen übernommen. Falls die beiden Spannen sich nicht überschneiden, wird das Pairing abgebrochen.

\paragraph{Pairing: Phase 2} \mbox{} \vspace{0.2cm} \\
Anhand der ausgetauschten Informationen aus der ersten Phase wird in der zweiten Phase entschieden, welche der folgenden Methoden zur Generierung des Short Term Key (STK) bzw. Long Term Key (LTK) zu verwenden ist:
\begin{itemize}
    \item{Numeric Comparison (für LE erst ab Bluetooth-Version 4.2)}
    \item{Just Works}
    \item{Out of Band (OOB)}
    \item{Passkey Entry}
\end{itemize}
Da das Pairing (speziell Phase 2) für LE in der Bluetooth-Version 4.0 funktionale Unterschiede zum Pairing für LE ab Version 4.2 aufweist, wird das Pairing für LE in Version 4.0 als \textbf{LE Legacy Pairing} und das Pairing für LE ab Version 4.2 als \textbf{LE Secure Connections Pairing} bezeichnet. Aus Sicht des Nutzers sind beide Verfahren jedoch gleich und jede Bluetooth-Version, die LE unterstützt, unterstützt das LE Legacy Pairing. Im Gegensatz zum LE Secure Connections Pairing bietet LE Legacy Pairing für die Methoden Just Works und Passkey Entry keinen Schutz vor passivem Abhören während des Pairings, da LE Secure Connections Elliptic Curve Diffie-Hellman (ECDH) für den Schlüsselaustausch nutzt und LE Legacy Pairing nicht. \cite{BtSpec4.2_248}
% QUELLE Spec 4.2 S. 248 5.4.1 Association Models
\\\\
Verfügen beide Geräte über OOB-Authentifizierungsdaten für das LE Legacy Pairing, wird unabhängig vom jeweiligen MITM-Flag die Methode OOB gewählt. In LE Secure Connections Pairing wird ebenso verfahren, nur dass hier nicht die OOB-Flags beider Geräte gesetzt sein müssen (aber können), da ein gesetztes OOB-Flag genügt. Sind die MITM-Flags beider Geräte nicht gesetzt, wird die Methode Just Works ausgeführt. Anderenfalls werden die Ein- und Ausgabemöglichkeiten der Geräte entsprechend \cite{BtSpec4.2_tab_2302-2303} in die Wahl der Methode einbezogen.
% Verweis TABELLE Spec 4.2 S. 2302 f., Table 2.8

\subparagraph{Pairing Methoden} \mbox{} \vspace{0.2cm} \\
Bei der \textbf{Numeric Comparison} wird dem Nutzer auf beiden Geräten jeweils eine zufällig generierte sechsstellige Dezimalzahl angezeigt. Diese muss der Nutzer vergleichen und im Falle der Übereinstimmung auf beiden Geräten bestätigen oder anderenfalls ablehnen. Somit kann der Nutzer unabhängig von der Namensgegebung der Geräte sicherstellen, die richtigen Geräte ausgewählt zu haben. Zudem bietet diese Methode Schutz vor MITM-Angriffen. Außenstehende, die Kenntnis über diese Zahl gewonnen haben, können laut \cite{BtSpec4.2_244-245} damit keinen Vorteil zur Entschlüsselung der zwischen den beiden Geräten ausgetauschten Daten erlangen, da die Zahl nicht als Eingabe zur Generierung eines Schlüssels verwendet wird.
% QUELLE VERWEIS Sepc 4.2 S. 244 f. 5.2.4.1 Numeric Comparison

\textbf{Just Works} basiert auf der Funktionsweise von Numeric Comparison mit dem Unterschied, dass hier dem Nutzer keine sechsstellige Dezimalzahl ausgegeben wird und er die Verbindung nur bestätigen muss. Dadurch bietet Just Works keinen Schutz vor MITM-Angriffen, aber einen Schutz gegen passives Abhören (außer für LE Legacy Pairing). \cite{BtSpec4.2_245}
% QUELLE Spec 4.2 S.245 5.2.4.2 Just Works

\textbf{Out of Band} ist die Nutzung einer anderen Technologie (z.B. Near Field Communication), um Geräte zu entdecken oder um kryptographische Informationen für den Pairing-Prozess auszutauschen. Dabei sollte die Technologie Schutz vor MITM-Angriffen bieten. \cite{BtSpec4.2_246}
% QUELLE Spec 4.2 S. 246 5.2.4.3 Out of Band

Die Methode \textbf{Passkey Entry} definiert, dass ein Gerät die zufällig generierte sechstellige Dezimalzahl ausgibt und der Nutzer sie auf dem anderen Gerät eingeben muss. Ein Schutz gegen MITM-Angriffe ist gegeben, da diese nur mit einer Wahrscheinlichtkeit von 0,0001\% für jede Durchführung der Methode möglich sind. Schutz gegen passives Abhören bietet Passkey Entry nur in LE Secure Connections Pairing und nicht in LE Legacy Pairing. \cite{BtSpec4.2_246-247} \cite{BtSpec4.2_2304}
% QUELLE Spec 4.2 S. 246 f. 5.2.4.4 Passkey Entry
% QUELLE Spec. 4.2 S. 2304 2.3.5.3 LE Legacy Pairing - Passkey Entry

\subparagraph{LE Legacy Pairing: Schlüssel und deren Generierung} \mbox{} \vspace{0.2cm} \\
Beim LE Legacy Pairing zweier Geräte generieren beide einen 128 Bit langen Temporary Key (TK), der bei der Authentifizierung genutzt wird, um den STK zu generieren und die Verbindung zu verschlüsseln. In Just Works wird der TK auf null gesetzt. Bei der Methode Passkey Entry ist der TK die besagte zufällig generierte sechstellige Dezimalzahl, die bereits mit 20 Bit dargestellt werden kann, weswegen die restlichen Bit des TK auf null gesetzt werden müssen. Dagegen kann bei OOB auf diese Einschränkung verzichtet werden, wodurch der TK tatsächlich eine Länge von 128 Bit aufweist.
\\\\
Das Gerät, welches das Pairing einleitet (Master), generiert eine zufällige 128 Bit lange Nummer \textit{Mrand} und ermittelt den gleichlangen Bestätigungswert \textit{Mconfirm} mit der Confirm Value Generation Function c1 \cite{BtSpec4.2_2288}. 
% QUELLE Spec. 4.2 S. 2288 2.2.3 Confirm value generation function c1 for LE Legacy Pairing
Zur Berechnung von \textit{Mconfirm} erhält die Funktion c1 folgende Eingabewerte entsprechend Gl. \ref{eq: mconfirm} \cite{BtSpec4.2_2305-2306}:
\begin{equation}
\begin{split}
    \text{Mconfirm} = \text{c1}(& \text{TK, Mrand,} \\
    & \text{Pairing Request Command, Pairing Response Command,} \\
    & \text{Adresstyp des Masters, Adresse des Masters,} \\
    & \text{Adresstyp des Slaves, Adresse des Slaves}) \\
\end{split}
    \label{eq: mconfirm}
\end{equation}
Ebenso führt das antwortende Gerät (Slave) diese Schritte durch, wobei \textit{Mrand} als \textit{Srand} bezeichnet wird und \textit{Mconfirm} als \textit{Sconfirm}. Die Eingabewerte der Funktion c1 zur Berechnung von \textit{Sconfirm} sind demnach analog zu denen von \textit{Mconfirm}. Danach findet gemäß Abb. \ref{fig: austausch vor stk generierung} folgender Austausch statt:

\begin{figure}[H]
    \centering
    \includegraphics[width=0.9\textwidth]{graphics/austausch_vor_stk_generierung.pdf}
    \caption[Austausch von \textit{Mconfirm}, \textit{Sconfirm}, \textit{Mrand} und \textit{Srand} zwischen Master und Slave]{Austausch von \textit{Mconfirm}, \textit{Sconfirm}, \textit{Mrand} und \textit{Srand} zwischen Master und Slave \cite{BtSpec4.2_2305-2306}}
    \label{fig: austausch vor stk generierung}
\end{figure}

Master und Slave tauschen \textit{Mconfirm} und \textit{Sconfirm} aus. Danach überträgt der Master \textit{Mrand}, damit der Slave \textit{Mconfirm} entsprechend Gl. \ref{eq: mconfirm} berechnen und so den empfangenen \textit{Mconfirm} verifizieren kann. Nach einer erfolgreichen Prüfung von \textit{Mconfirm} überträgt der Slave \textit{Srand}, damit der Master \textit{Sconfirm} analog zu Gl. \ref{eq: mconfirm} berechnen und somit den empfangenen \textit{Sconfirm} verifizieren kann. Wenn einer der Bestätigungswerte (\textit{Mconfirm}, \textit{Sconfirm}) nicht erfolgreich verifiziert wird, wird die Verbindung sofort beendet.
\\\\
Anschließend wird der STK mit der Funktion s1 \cite{BtSpec4.2_2290} 
% QUELLE verweis auf Spec 4.2 S. 2290 2.2.4 Key generation function s1 for LE Legacy Pairing
entsprechend Gl \ref{eq: stk} \cite{BtSpec4.2_2305-2306} berechnet.

\begin{equation}
    \text{STK} = \text{s1}(\text{TK, Srand, Mrand})
    \label{eq: stk}
\end{equation}

Demnach kann bei der Methode Passkey Entry kein ausreichender Schutz gegen passives Abhören geboten werden, da der TK nur wenig mögliche Werte annehmen kann. Ist die vereinbarte Schlüsselgröße kleiner als 128 Bit, werden die überschüssigen Bit beginnend bei dem MSB auf null gesetzt. Der STK wird nun zur Verschlüsselung der Verbindung genutzt. \cite{BtSpec4.2_2305-2306}
% QUELLE Spec. 4.2 S. 2305 f. 2.3.5.5 LE Legacy Pairing Phase 2

\subparagraph{LE Secure Connections Pairing: Schlüssel und deren Generierung} \mbox{} \vspace{0.2cm} \\
Beim LE Secure Connections Pairing wird ein Long Term Key (LTK) erstellt. Zuvor generieren beide Geräte jeweils ein ECDH-Schlüsselpaar (PK - Public Key, SK - Private Key) und tauschen ihre Public Keys aus. Danach berechnet jedes Gerät den Diffie-Hellman-Schlüssel aus seinem Private Key und dem Public Key des Gegenübers. Durch den Diffie-Hellman-Schlüssel kennen beide Parteien ein gemeinsames Geheimnis, mit dem sie den weiteren Datenaustausch zur Authentifizierung verschlüsseln können. Die Authentifizierung ist notwendig, da der ECDH-Schlüsselaustausch zwar resistent gegen passives Abhören ist, jedoch nicht gegen MITM-Angriffe. \cite{BtSpec4.2_2307}
% QUELLE Spec. 4.2 S. 2307 2.3.5.6.1 Public Key Exchange

Diese Authentifizierung wird mit den Pairing Methoden Numeric Comparison, Just Works, OOB und Passkey Entry ermöglicht. Jedoch unterscheiden diese sich aus funktionaler Sicht (nicht aus Nutzersicht) vom LE Legacy Pairing durch komplexere Verfahren. Letztendlich lässt sich für die vier Pairing-Methoden Folgendes zusammenfassen: 
\begin{itemize}
    \item Numeric Comparison signalisiert dem Nutzer mit einer Wahrscheinlichkeit von 99,9999\% einen stattfindenden MITM-Angriff \cite{BtSpec4.2_2309}
% QUELLE Spec. 4.2 S. 2309, 2.3.5.6.2 Authentication Stage 1 – Just Works or Numeric Comparison
    \item Just Works bietet keinen Schutz vor einem MITM-Angriff \cite{BtSpec4.2_245} 
% QUELLE Spec. 4.2 S. 245, 5.2.4.2 Just Works
    \item Ein MITM-Angriff während des Passkey Entry gelingt nur mit einer Wahrscheinlichkeit von 0,0001\% \cite{BtSpec4.2_2311}
% QUELLE Spec. 4.2 S. 2311, 2.3.5.6.3 Authentication Stage 1 – Passkey Entry
    \item Anfälligkeit auf Angriffe bei der OOB-Methode ist von der verwendeten OOB-Technologie abhängig \cite{BtSpec4.2_2312-2313}
% QUELLE Spec. 4.2 S. 2312 f., 2.3.5.6.4 Authentication Stage 1 – Out of Band
\end{itemize}

Nach der Ausführung einer Pairing-Methode wird der LTK als Teilergebnis der Funktion f5 \cite{BtSpec4.2_2292-2293} 
% QUELLE Spec. 4.2 S. 2292 f., 2.2.7 LE Secure Connections Key Generation Function f5
mit den Eingabewerten Diffie-Hellman-Key, einer Nonce des Masters, einer Nonce des Slaves und der Adresse des Masters und Slaves ermittelt \cite{BtSpec4.2_2314}.
% QUELLE Spec. 4.2 S. 2314, 2.3.5.6.5 Authentication Stage 2 and Long Term Key Calculation

\paragraph{Pairing: Phase 3} \mbox{} \vspace{0.2cm} \\
Wurde der STK bzw. LTK generiert, wird er genutzt, um die Verbindung zu verschlüsseln. Nun können in der dritten Phase transportspezifische Schlüssel ausgetauscht werden. Z.B. wird der Identity Resolving Key (IRK) zur Generierung und Auflösung von zufälligen Adressen verwendet und der Connection Signature Resolving Key (CSRK) zur Signatur von Daten und Überprüfung von Signaturen.

% Für Numeric Comparison und Just Works wird entsprechend Abb. X fortgefahren.
% % TODO BILD VERWEIS

% \begin{figure}[hbt!]
%     \centering
%     \inlcudegraphics[width=0.5\linewidth]{graphics/LE_Secure_Connections_Pairing_Numeric_Comparison_Just_Works.svg}
%     \caption{}
% \end{figure}

% Der zu Beginn zufällig generierte Nonce (N\textunderscore a bzw. N\textunderscore b) mit einer Länge von 128 Bit wird für jeden Durchlauf neu erzeugt und schützt vor Replay-Angriffen. Für die Berechnung der Bestätigung C\textunderscore b wird die Einwegfunktion f4 [X] genutzt. 
% % TODO QUELLE Spec. 4.2 S. 2291 2.2.6 LE Secure Connections Confirm Value Generation Function f4
% Mittels der Funktion g2 [X] 
% % TODO QUELLE Spec 4.2 S. 2295 2.2.9 LE Secure Connections Numeric Comparison Value Generation Function g2
% werden die Dezimalzahlen D\textunderscore a und D\textunderscore b berechnet. Darauf werden diese dem Nutzer auf den Geräten ausgegeben und dieser muss deren Gleichheit bestätigen. Wird anstatt Numeric Comparison die Methode Just Works ausgeführt, dann werden D\textunderscore a und D\textunderscore b nicht berechnet und folglich nicht dem Nutzer gezeigt. Sollte ein Fehler auftreten wird das Protokoll abgebrochen und kann neu gestartet werden.

% Beim vorherigen ECDH-Schlüsselaustausch, kann ein MITM-Angriff angewandt werden. Eine einfache Variante, die keine Informationen der beiden angegriffenen Geräte zwischen diesen austauscht sondern mit diesen jeweils separat die Methode Numeric Comparison durchführt, endet darin, dass dem angegriffenem Nutzer mit einer Wahrscheinlichkeit von 0,999999 auf dessen Geräten zwei verschiedene Dezimalzahlen angezeigt werden. Eine andere Variante wäre aus Sicht des Angreifenden den Verkehr bestehend aus der Bestätigung C\textunderscore b, dem Nonce N\textunderscore a und N\textunderscore b nur weiterzuleiten. Jedoch wird der Master bei der Überprüfung der Bestätigung C\textunderscore b feststellen, dass diese nicht mit seinem Public Key PK\textunderscore a erstellt wurde, was zum Abbruch führt.\\\\
% % TODO QUELLE Spec. 4.2 S. 2308 f. 2.3.5.6.2 Authentication Stage 1 – Just Works or Numeric Comparison
% Passkey Entry
% OOB

	\subsection{Sicherheit}
		\label{sec: le security}
		LE association models
    - MITM Chance 1/1'000'000

GAP LE Security Aspects S. 1722 (Privacy feature, Random Device Address, ...)

Application Layer Security bei Smartphones ?

BLE Sicherheit abhängig von Eingabe- und Ausgabemöglichkeiten

quellen aus bookmarks prüfen

\newpage
\section{Infrastruktur}
	\label{sec: infra allg}
	Die Infrastruktur beschreibt eine allgemeine Lösung, um eine sichere Kommunikation mittels Bluetooth Low Energy zu gewährleisten. Bis auf den Fakt, dass ein Mikrocontroller und ein Smartphone miteinander kommunizieren, sind die Systeme der beiden Kommunikationsparteien nicht von Relevanz. Dabei ist die Lösung auf weitere Konstellationen der Systeme anwendbar, solang jedes System über ein Bluetooth-Modul (mind. der Bluetooth-Version 4.0) verfügt und eine Software-Bibliothek für Transport Layer Security (TLS) der Version 1.2 oder höher unterstützt. Somit ist die Infrastruktur unabhängig von dem in der Sektion X beschriebenen Projekt SteigtUM.
% TODO SEKTION VERWEIS auf Einleitung
\\\\
Obwohl die Sicherheit in der Datenübertragung keine allumfassende Definition besitzt, lassen sich für diese trotzdem essenzielle Aspekte formulieren. Nach X 
% TODO QUELLE S. 19 https://books.google.de/books?id=-fciBAAAQBAJ&lpg=PA2&ots=YPh7AK5_WJ&dq=sichere%20Daten%C3%BCbertragung&lr&pg=PA19#v=onepage&q&f=false
sind diese Vertraulichkeit, Datenintegrität und Authentzität.

Vertraulichkeit bedeutet, dass "Übertragene Daten [...] nur berechtigten Instanzen zugänglich sein [sollen], d.h. keine unbefugte dritte Partei soll an den Inhalt von übertragenen Nachrichten gelangen können".
% TODO QUELLE ZITAT kapitel 2.6 Sicherheitsziele in Netzwerken S. 19 https://books.google.de/books?id=-fciBAAAQBAJ&lpg=PA2&ots=YPh7AK5_WJ&dq=sichere%20Daten%C3%BCbertragung&lr&pg=PA19#v=onepage&q&f=false

Die Datenintegrität trägt folgende Bedeutung. "Für den Empfänger muss eindeutig erkennbar sein, ob Daten während ihrer Übertragung unbefugt geändert wurden".
% TODO QUELLE ZITAT kapitel 2.6 Sicherheitsziele in Netzwerken S. 19 https://books.google.de/books?id=-fciBAAAQBAJ&lpg=PA2&ots=YPh7AK5_WJ&dq=sichere%20Daten%C3%BCbertragung&lr&pg=PA19#v=onepage&q&f=false

Authentizität teilt sich in zwei Punkte auf. Zum einen soll "Eine Instanz [...] einer an deren ihre Identität zweifelsfrei nachweisen können (Identitätsnachweis bzw. Authentifizierung der Instanz)". 
% TODO QUELLE ZITAT kapitel 2.6 Sicherheitsziele in Netzwerken S. 19 https://books.google.de/books?id=-fciBAAAQBAJ&lpg=PA2&ots=YPh7AK5_WJ&dq=sichere%20Daten%C3%BCbertragung&lr&pg=PA19#v=onepage&q&f=false
Zum anderen "soll überprüft werden können, ob eine Nachricht von einer bestimmten Instanz stammt (Authentizität der Daten)".
% TODO QUELLE ZITAT kapitel 2.6 Sicherheitsziele in Netzwerken S. 19 https://books.google.de/books?id=-fciBAAAQBAJ&lpg=PA2&ots=YPh7AK5_WJ&dq=sichere%20Daten%C3%BCbertragung&lr&pg=PA19#v=onepage&q&f=false

	\subsection{Topologie}
		\label{sec: infra topologie}
		Die Topologie entspricht der vorgestellten Topologie der Infrastruktur (siehe Sektion \ref{sec: infra topologie}). Zusätzlich wird ein Back End in Form eines Servers benötigt, um den autonomen Verleih zu steuern. Die Abb. \ref{fig: impl topo} zeigt die Topologie der Implementierung.
\begin{figure}[H]
    \centering
    \includegraphics[width=1\textwidth]{graphics/impl_topologie.pdf}
    \caption[Topologie der Implementierung]{Topologie der Implementierung}
    \label{fig: impl topo}
\end{figure}
Bevor ein Ausleihprozess stattfinden kann, müssen Back End, Mikrocontroller und Smartphone jeweils über ein von der Zertifizierungsstelle ausgestelltes Zertifikat verfügen. Jeder Partei außer der Zertifizierungsstelle sollte periodisch (z.B. jährlich) ein Zertifikat ausgestellt werden.

	\subsection{Transport}
		\label{sec: infra transport}
		Eine der grundlegendsten Bedingungen dieser Arbeit ist, dass BLE als Technologie zum Übertragen der Daten zwischen Smartphone und Mikrocontroller genutzt werden soll. Demnach stellt sich die Frage nach einem geeigneten Transport der Daten innerhalb der BLE"=Architektur.
\\\\
Es existieren mehrere Referenzmodelle für die Kommunikation innerhalb von Rechnernetzen. Geläufig sind das TCP/IP"=Referenzmodell (Transport Control Protocol / Internet Protocol), das OSI"=Referenzmodell (Open Systems Interconnection) und ein hybrides Referenzmodell aus diesen beiden. Alle drei legen sich auf unterschiedliche Anzahlen von Schichten fest, von denen sich einige gleichen oder ähneln und andere nicht. Die Eigenschaften der Transportschicht sind bei den drei Referenzmodellen identisch. Die zu übertragenden Daten der darüberliegenden Anwendungsschicht werden in Segmente aufgeteilt und von einem Endpunkt über die niedrigeren Schichten zum anderen Endpunkt gesendet. Auf Empfängerseite erhält die Transportentität diese Segmente, setzt sie wiederzusammen und übergibt sie der zugehörigen Anwendung, die mithilfe von Ports identifiziert wird. Demnach kann die Transportschicht eine verbindungslose oder verbindungsorientierte Datenübertragung unterstützen, wobei die verbindungsorientierte Übertragung Flusskontrolle, verlustfreie Übertragung und die korrekte Reihenfolge der Segmente unterstützen kann. \cite{Baun2019_36-40}
% QUELLE https://katalog.ub.tu-freiberg.de/Record/0-1666728691, https://link.springer.com/book/10.1007%2F978-3-658-26356-0, Computer Networks / Computernetze von Christian Baun, S. 39 Transportschicht
\\\\
Die BLE"=Architektur lässt sich entsprechend der Abb. \ref{fig: hyb referenzmodell ble} am besten dem hybriden Referenzmodell zuordnen, da dieses im Gegensatz zum OSI-Referenzmodell die Anwendungsschicht nicht in drei weitere Schichten unterteilt und abgesehen von dieser Differenz dem OSI-Referenzmodell gleicht.

\begin{figure}[H]
    \centering
    \includegraphics[width=0.55\textwidth]{graphics/hybr_referenzmodell_zu_ble.pdf}
    \caption[Zuordnung der BLE Layer zum hybriden Referenzmodell]{Zuordnung der BLE Layer zum hybriden Referenzmodell}
    \label{fig: hyb referenzmodell ble}
\end{figure}

Der Physical Layer bzgl. BLE stimmt mit dem Physical Layer des hybriden Referenzmodells überein, da hier die physikalische Bitübertragung definiert wird. Der Link Layer bzgl. BLE unterstützt wie der Link Layer des hybriden Referenzmodells den Zugriff auf das Übertragungsmedium mittels der Access Address und das Erkennen von fehlerhaft übertragenen Paketen (BLE) bzw. Frames (hybrides Referenzmodell) mittels einer Prüfsumme. Auch physische Adressen (zumindest die öffentliche Bluetooth"=Adresse) finden sich im Link Layer von BLE beim Advertising und Verbindungsaufbau wieder. Ein äquivalent zum Network Layer des hybriden Referenzmodells existiert nicht, da nur Punkt"=zu"=Punkt"=Verbindungen existieren, wodurch ein Routing der Pakete nicht nötig ist.
\\\\
L2CAP bzw. der L2CAP Layer kann als Äquivalent zum Transport Layer des hybriden Referenzmodells angesehen werden. Die Nutzdaten der Anwendungsschicht werden in Segmente aufgeteilt und über die niedrigeren Schichten an den L2CAP Layer des Empfängers übertragen. Über die L2CAP Channels und deren CIDs können diese Nutzdaten den Anwendungen zugeordnet werden. L2CAP ist in der Lage die Frames/PDUs verbindungsorientiert oder verbindungslos zu kommunizieren. Im LE Credit Based Flow Control Mode unterstützt BLE ab Bluetooth"=Version 4.2 Flusskontrolle. Eine verlustfreie Übertragung wird bereits im Link Layer bereitgestellt, da jedes Link"=Layer"=Paket vom Empfänger positiv (ACK) oder negativ (NAK) bestätigt wird. Wird ein Paket negativ bestätigt, wird es erneut gesendet.

Alle weiteren Protokolle wie GAP, SMP und GATT nutzen den Transport durch L2CAP. GAP dient dem Verbindungsaufbau und SMP bietet Sicherheitsfunktionen. Sie gehören im hybriden Referenzmodell dem "`oberen Teil"' des Transport Layers an.
\\\\
Ein wichtiger Schwerpunkt von BLE liegt in der Übertragung von Sensordaten \cite{BtDataTransfer}. Da GATT für die Übertragung von Sensordaten geeignet ist, tritt es häufig bei der Recherche nach der Funktionsweise und den Anwendungen von BLE auf. Jedoch basiert GATT seinem Namen entsprechend auf hierarchisch gegliederten Attributen, die unteranderem veränderbare Werte beinhalten. Ein Server erstellt solche Attribute und gibt diese über das Advertising oder erst über eine Verbindung zu einem Client bekannt. Der Client kann die Werte der Attribute mittels Anfragen lesen und schreiben. Der Server verfügt über die Attribute, wesewegen er diese zu jeder Zeit lesen und schreiben kann. Zudem kann der Server Attribute/Werte direkt an einen Client senden ohne eine vorherige Anfrage von diesem erhalten zu haben. 

Ein geeigneter Anwendungsfall wäre demnach ein Server, der über einen oder mehrere Sensoren verfügt, deren Ausgabewerte er zeitlich periodisch in seine vorher erstellten Attribute schreibt. Ein oder mehrere Clients können dann über die Attribute spezifische Sensordaten anfragen und empfangen.
\\\\
Daraus kann gefolgert werden, dass GATT im Gegensatz zum reinen L2CAP einen geringen Overhead benötigt und aufgrund seiner Architektur für diese Infrastruktur als Transportprotokoll eher ungeeignet ist. Trotzdem sei dem anzufügen, dass die meisten BLE"=Programmierschnittstellen den Zugang zu GATT ermöglichen.

Der Zugang zu L2CAP ist dagegen in einigen BLE"=Programmierschnittstellen nicht verfügbar. Ein Beispiel dafür sind die beiden gängigen Betriebssysteme \textit{Android} und \textit{iOS}. \textit{Android} unterstützt BLE allgemein seit Android 4.3 \cite{AndroidAppLayerSec}, doch ermöglicht den Zugang zu L2CAP (also LE Connection"=Oriented Channels) erst ab Android 10 \cite{AndroidCoC}. \textit{iOS} unterstützt BLE allgemein seit iOS 5.0 \cite{iOS_coreBluetooth} und ermöglicht den Zugang zu L2CAP erst ab iOS 11.0 \cite{iOS_CBL2CAPChannel}.
\\\\
Trotz des eingeschränkten Zugangs zu L2CAP in vielen Programmierschnittstellen, wird es als Transportprotokoll für diese Infrastruktur gewählt. Die Nutzung von GATT würde einen geringen Overhead erzeugen, da es ohnehin auf L2CAP basiert, und bis auf einen besseren Zugang keine weiteren Vorteile bieten.

	\subsection{Sicherheit}
		\label{sec: infra sicherheit}
		LE association models
    - MITM Chance 1/1'000'000

GAP LE Security Aspects S. 1722 (Privacy feature, Random Device Address, ...)

Application Layer Security bei Smartphones ?

BLE Sicherheit abhängig von Eingabe- und Ausgabemöglichkeiten

quellen aus bookmarks prüfen

	\subsection{Verbindungsaufbau}
		\label{sec: infra verbindungsaufbau}
		Um mittels BLE ein Piconet zu bilden, benötigt es einen Advertiser. Auf drei vorgegebenen Frequenzen, den Advertising Channels (siehe X)
% TODO SEKTION VERWEIS grundlagen ble controller physical channel
, sendet dieser Daten, mit denen er sich für andere Geräte bemerkbar macht (Advertisements). Dabei können Advertisements auch genutzt werden, um Nutzdaten zu senden. Jedes Advertisement-Paket beinhaltet eine Bluetooth-Adresse des Senders, welche 48 Bit lang ist.

Geräte, die Daten auf den Advertising Channels empfangen, werden Scanner bzw. Initiator genannt. Auf diesem Weg finden sich die Geräte (Discovering). Der Initiator unterscheidet sich vom Scanner, da er in der Lage ist, sich zu einem Advertiser zu verbinden, von dem er ein Advertisement erhielt, dass das Verbinden zu diesem ermöglicht. Sind zwei Geräte verbunden senden und empfangen sie ihre Pakete auf den Data Channels (siehe X).
% TODO SEKTION VERWEIS grundlagen ble controller physical channel
Verbinden sich zwei Geräte wird der Initiator zum Master und der Advertiser zum Slave.
\\\\
Durch die Anwendung von Zeitmultiplexing senden die Geräte ihre Pakete immer zu festgelegten Zeitpunkten. Dabei ist ein Event ein zeitlicher Abschnitt, in dem zusammenhängende Daten in Form von Paketen gesendet bzw. empfangen werden.
% TODO BILD lies folgenden Satz, siehe auch Spec S. 127 oben
In Abbildung X ist ein Advertising Event dargestellt bei dem ein Advertiser auf allen drei Advertising Channels nacheinander Advertisement-Pakete sendet. Auf dem zweiten Kanal empfängt der Advertiser direkt gefolgt auf sein erstes Advertisement-Paket in diesem Kanal ein Paket eines Scanners, auf welches er mit einem weiteren Advertisement antwortet.
% TODO BILD Spec S. 127 unteres bild
In Abbildung X ist ein Advertising Event dargestellt, bei dem ein Initiator auf das Advertisement-Paket eines Advertiser antwortet, um eine Verbindung aufzubauen. Darauf folgt ein Connection Event bei dem Master (ursprünglich Initiator) und Slave (ursprünglich Advertiser) auf einem Data Channel sich gegenseitig Pakete senden. Danach folgt ein weiteres Connection Event auf einem anderen Data Channel.

\newpage
\section{Implementierung}
	\label{sec: impl allg}
	Mithilfe der definierten Infrastruktur kann nun eine Implementierung für einen spezifischen Anwendungsfall erstellt werden. In dieser Arbeit wird die Infrastruktur für einen autonomen Verleih von elektrischen Kleinfahrzeugen in Bezug auf das Projekt \textit{SteigtUM} (siehe Sektion \ref{sec: einleitung}) angewandt. Der Verleihdienst sieht vor, dass ein Benutzer mit einem Smartphone ein Kleinfahrzeug ausleihen kann. In Bezug auf diese Arbeit sollen Smartphone und Kleinfahrzeug (Mikrocontroller) mittels BLE miteinander kommunizieren. Um einen Ausleihprozess einzuleiten wird ein Back End in Form eines Servers benötigt, damit ein Verwaltungsmodell bzgl. Zahlungen realisiert werden kann. Dieses Modell ist jedoch nicht Bestandteil der Arbeit. Dennoch wird für den Prototyp (und für die spätere reale Implementierung) ein Back End benötigt, um den Benutzer bzw. die Smartphone-Anwendung (App) zum Ausleihen eines Fahrzeugs zu autorisieren.
\\\\
Innerhalb des Prototyps kommunizieren anstelle eines Smartphones und eines Mikrocontrollers vorerst nur zwei Mikrocontroller. Ein Mikrocontroller wird mit dem Verleihfahrzeug assoziiert, während der andere das Smartphone des Benutzers vertritt. Die weiteren Kommunikationsparteien Back End und Zertifizierungsstelle werden nur simuliert.
\\\\
Für die Implementierung wurden einige Bedingungen aufgestellt, damit sie das konkrete Modell eines Verleihdienstes nicht einschränkt. Eine Bedingung ist, dass zu Beginn eines Ausleihprozesses die App mit dem Back End (sicher) kommunizieren kann, bis der Benutzer das Fahrzeug nutzen kann. Danach darf die Implementierung keine Verbindung zwischen App und Back End weiterhin voraussetzen. Zwischen dem Fahrzeug und dem Back End darf nie eine Verbindung vorausgesetzt werden. Wird die Bluetooth-Verbindung zwischen App und Fahrzeug unterbrochen, darf nicht vorausgesetzt werden, dass zwischen App und Back End eine (sichere) Verbindung hergestellt werden kann. Somit bleibt die Implementierung weitestgehend unabhängig von äußeren Einflüssen, die die Verbindungen zum Back End beeinträchtigen können. Außerdem gibt die Implementierung nicht vor, welche technischen Voraussetzungen das Fahrzeug für die Kommunikation benötigt (mit Ausnahme von BLE).

	\subsection{Topologie}
		\label{sec: impl topologie}
		Die Topologie entspricht der vorgestellten Topologie der Infrastruktur (siehe Sektion \ref{sec: infra topologie}). Zusätzlich wird ein Back End in Form eines Servers benötigt, um den autonomen Verleih zu steuern. Die Abb. \ref{fig: impl topo} zeigt die Topologie der Implementierung.
\begin{figure}[H]
    \centering
    \includegraphics[width=1\textwidth]{graphics/impl_topologie.pdf}
    \caption[Topologie der Implementierung]{Topologie der Implementierung}
    \label{fig: impl topo}
\end{figure}
Bevor ein Ausleihprozess stattfinden kann, müssen Back End, Mikrocontroller und Smartphone jeweils über ein von der Zertifizierungsstelle ausgestelltes Zertifikat verfügen. Jeder Partei außer der Zertifizierungsstelle sollte periodisch (z.B. jährlich) ein Zertifikat ausgestellt werden.

	\subsection{Ausleihprozess}
		\label{sec: impl ausleih allg}
		Mit der Anwendung der Infrastruktur für die Implementierung wird bereits der Aufbau einer sicheren Verbindung zwischen Smartphone und Mikrocontroller ermöglicht. Ab diesen Punkt muss der Mikrocontroller sicherstellen, dass der Benutzer bzw. das Smartphone dazu autorisiert ist, das Fahrzeug auszuleihen. Dieses Problem wird mit einer vom Back End unterzeichneten Bestätigung gelöst, die im Rahmen dieser Arbeit als "`Subscription"' bezeichnet wird.
\\\\
Die Subscription muss folgende Anforderungen erfüllen. Sie muss beweisen, dass sie ausschließlich vom Back End angefertigt wurde (Authentizität). Falls ihr Inhalt von einer anderen Instanz verändert wurde, muss dies bei der Prüfung der Subscription durch den Mikrocontroller ersichtlich werden (Datenintegrität). Die Informationen, die sie enthält, beziehen sich auf ihren zugehörigen Ausleihvorgang (z.B. Identität des Fahrzeugs/Mikrocontrollers, Zeitstempel).

		\subsubsection{Verbindungsaufbau und Autorisierung}
			\label{sec: impl verbindungsaufbau und autorisierung}
			Bevor die Subscription erstellt bzw. vom Smartphone angefordert wird, herrscht folgende Ausgangssituation. Vor der Inbetriebnahme des Verleihdienstes müssen Smartphone, Back End und Mikrocontroller jeweils über ein individuelles Zertifikat (nach X.509 Standard) und dessen privaten Schlüssel verfügen. Das Zertifikat wurde jeweils von der Zertifizierungsstelle ausgestellt. Zudem kennt auch jeder der drei Parteien (Smartphone, Back End, Mikrocontroller) das Root-Zertifikat (das Zertifikat der Zertifizierungsstelle), um damit die anderen Parteien authentifizieren zu können. Außerdem benötigt das Back End ein weiteres Zertifikat mit zugehörigem privaten Schlüssel, das Subscription-Zertifikat genannt wird.
\\\\
Nun möchte ein Nutzer mit der Smartphone-Anwendung (App) ein Fahrzeug ausleihen. Dafür sollte das Fahrzeug signalisieren, dass es für einen Ausleihvorgang zur Verfügung steht. Ab diesen Punkt ist der weitere Verlauf des Ausleihprozesses in Abb. \ref{fig: verlauf ausleihprozess} dargestellt.
\begin{figure}[H]
    \centering
    \includegraphics[width=1\textwidth]{graphics/verlauf_ausleihprozess.pdf}
    \caption[Verlauf des Ausleihprozzeses]{Verlauf des Ausleihprozzeses}
    \label{fig: verlauf ausleihprozess}
\end{figure}
Zu Beginn muss die App die Identität und Bluetooth-Adresse des Fahrzeugs feststellen (z.B. über einen Quick Response Code, kurz QR-Code). Die App baut eine sichere Verbindung zum Back End auf und verlangt nach einer Subscription. Dabei sendet sie die Identität des Fahrzeugs, einen aktuellen Zeitstempel und evtl. weitere Informationen (z.B. bzgl. des Bezahlmodells). Das Back End fügt zu diesen Informationen evtl. noch weitere hinzu (z.B. bzgl. des Bezahlmodells) und formt daraus den "`Payload"'. Danach wird der Hashwert des Payloads gebildet und mit dem privaten Schlüssel signiert. Die Subscription setzt sich nun entsprechend Abb. \ref{fig: aufbau subscription} zusammen.
\begin{figure}[H]
    \centering
    \includegraphics[width=0.9\textwidth]{graphics/aufbau_subscription.pdf}
    \caption[Aufbau der Subscription]{Aufbau der Subscription}
    \label{fig: aufbau subscription}
\end{figure}
Das erste Längenfeld gibt die Länge des darauffolgenden Payloads an. Danach folgt der Payload. Das zweite Längenfeld gibt die Länge der Signatur an, worauf die Signatur folgt. Das letzte Längenfeld steht für die Länge des Subscription-Zertifikats und ist gefolgt vom Subscription-Zertifikat. Schließlich überträgt das Back End die Subscription an die App und trennt die Verbindung.
\\\\
Entsprechend der Sektion \ref{sec: infra verbindungsaufbau} verbindet sich die App über BLE mit dem Fahrzeug und stellt mittels TLS eine sichere Verbindung her. Darauf beendet das Fahrzeug das Advertising. Anschließend sendet die App die Subscription an das Fahrzeug. Zuerst verifziert das Fahrzeug das Subscription-Zertifikat, indem es dessen Signatur mit dem Root-Zertifikat prüft. Danach verifiziert es die Signatur des Payloads mit der gleichen Hash-Funktion, die zur Erstellung der Signatur genutzt wurde, und dem öffentlichen Schlüssel des Subscription-Zertifikats. Sind die Verifikationen erfolgreich, konnte das Fahrzeug sicherstellen, dass die Subscription vom Back End erstellt wurde. Nun müssen noch die Inhalte des Payloads geprüft werden. Demnach muss die angegebene Identität mit der des Fahrzeugs übereinstimmen und der Zeitstempel aktuell sein. Auch kann im Fall, dass weitere Informationen angegeben wurden, deren Plausibilität geprüft werden und deren Informationsgehalt verarbeitet werden. Somit ist der erste Teil des Ausleihprozesses abgeschlossen. Sollte eine der Verifikationen fehlschlagen sowohl bei der Prüfung der Subscription als auch vorher auf Ebene von TLS, wird die Verbindung abgebrochen.

		\subsubsection{Erneuter Verbindungsaufbau}
			\label{sec: impl erneuter verbindungsaufbau}
			Ab diesen Punkt muss ein weiterer Fall für den Verlauf des Ausleihprozesses berücksichtigt werden: der Abbruch der Verbindung zwischen Smartphone und Fahrzeug sowie die darauffolgende Wiederherstellung der Verbindung. Der Verbindungsabbruch kann zum einen durch technische Einflüsse (z.B. Interferenzen) verursacht werden und nicht vom Benutzer beabsichtigt sein. Zum anderen ist es denkbar, dass der Nutzer sich für eine bestimmte Zeitspanne vom Fahrzeug entfernt, wodurch die Verbindung aufgrund der begrenzten Bluetooth-Reichweite abbricht. Evtl. sollte hier der Benutzer dem Fahrzeug signalisieren, dass er für eine bestimmte Zeitspanne abwesend ist. Danach begibt sich der Nutzer wieder zum Fahzeug und möchte es weiter nutzen.
\\\\
Wenn die Verbindung nun abbricht, beginnt das Fahrzeug mit dem Advertising. Sobald der Nutzer bzw. das Smartphone in Reichweite ist, verbindet sich die App per BLE mit dem Fahrzeug, da es die Bluetooth-Adresse des Fahrzeugs bereits kennt, und baut wie oben beschrieben eine sichere Verbindung auf. Nun sendet die App die alte Subscription erneut an das Fahrzeug. Das Fahrzeug verifiziert erneut die Subscription und vergleicht sie mit der Subscription, die es zu Beginn des Ausleihprozesses empfangen hat. Stimmt sie mit der ursprünglichen Subscription überein, kann das Fahrzeug wieder genutzt werden. Anderenfalls trennt das Fahrzeug die Verbindung und beginnt das Advertising.

		\subsubsection{Beenden des Ausleihprozesses}
			\label{sec: impl beenden des ausleihprozesses}
			Das Beenden des Ausleihprozesses ist mitunter abhängig vom Bezahlmodell des Verleihdienstes. Im Wesentlichen könnte die App dem Fahrzeug eine Nachricht senden, die besagt, dass der Ausleihprozess nun beendet wird. Zudem sollte auch der Fall berücksichtigt werden, wenn keine ordnungsgemäße Beendigung stattfindet. Dabei sollte sich das Fahrzeug nach einer bestimmten Zeitspanne für neue Ausleihprozesse automatisch freigegeben.

		\subsubsection{Einfluss der Infrastruktur}
			\label{sec: impl einfluss der infrastruktur}
			Da die Zertifikate der Parteien (Back End, App, Fahrzeug) von der Zertifizierungsstelle des Verleihdienstes ausgestellt wurden und die Parteien nur das Root-Zertifikat des Verleihdienstes nutzen, um das Zertifikat eines Gegenüber zu verifizieren, können nur die Parteien TLS-Verbindungen zueinander aufbauen. Jede außenstehende Instanz, die versucht eine Verbindung zu einer der Parteien (Back End, App, Fahrzeug) aufzubauen, wird bereits beim TLS-Handshake abgewiesen. Grund dafür ist, dass die außenstehende Instanz kein von der Zertifizierungsstelle ausgestelltes Zertifikat besitzt. Sollte sie doch an ein solches Zertifikat gelangen, muss sie beim TLS-Handshake immer noch beweisen, dass sie den zugehörigen privaten Schlüssel besitzt. Aus diesem Grund muss jede der Parteien (Back End, App, Fahrzeug) den privaten Schlüssel geheim halten. Das bedeutet auch, dass ein Angreifer nicht in der Lage sein darf, den privaten Schlüssel aus einer der Parteien auszulesen (z.B. bei physischen Zugang). Deshalb sollten Zertifikate und private Schlüssel in Keystores gespeichert werden.

	\subsection{Prototyp}
		\label{sec: impl prototyp allgemein}
		In diesem Kapitel werden Details zum Prototyp angegeben. Wie bereits erwähnt, ist der Prototyp nicht für eine Smartphone-Anwendung und einen Mikrocontroller umgesetzt, sondern vorerst nur für zwei Mikrocontroller (ein Mikrocontroller vertritt die Rolle der Smartphone-Anwendung).

		\subsubsection{Hardware und Software}
			\label{sec: impl soft hard}
			Zur Entwicklung des Prototyps bzw. Implementierung wurde neben einem Personal Computer (PC) folgende Hardware verwendet:

\begin{itemize}
    \item Mikrocontroller \textit{ESP32-WROOM-32}
    \item Smartphone mit Android 11.0
\end{itemize}

Obwohl das Smartphone vorerst nicht Teil des Prototyps ist, wird es bzw. das Betriebssystem \textit{Android} einbezogen.

\subsubsection{ESP32-WROOM-32 (Mikrocontroller)}
Der \textit{ESP32-WROOM-32} unterstützt Bluetooth 4.2 für BR/EDR und BLE \cite{ESP32_6}. Der Hersteller \textit{Espressif} stellt das \textit{Espressif Internet of Things Development Framework} (ESP-IDF) \cite{ESPIDF}, mit dem Anwendungen für den \textit{ESP32-WROOM-32} entwickelt werden können. Das ESP-IDF unterstützt unter anderem folgende Software-Komponenten:

\begin{itemize}
    \item \textit{Free Real Time Operating System} (FreeRTOS)
    \item \textit{Serial Peripheral Interface Flash Files System} (SPIFFS)
    \item \textit{nimBLE}
    \item \textit{mbedTLS}
\end{itemize}

FreeRTOS dient als Betriebssystem für den \textit{ESP32-WROOM-32}. SPIFFS wird benötigt, um Dateien auf den \textit{ESP32-WROOM-32} zu "`flashen"' bzw. um auf diese während der Laufzeit zugreifen zu können. So können die Zertifikate und privaten Schlüssel vor der Laufzeit auf den \textit{ESP32-WROOM-32} übertragen werden.

\textit{nimBLE} ist eine Software-Bibliothek für den BLE-Host von der \textit{Apache Software Foundation} und bietet Programmierschnittstellen für GAP und L2CAP \cite{nimBLE}. Für GAP und L2CAP wird jeweils ein Event Handling ausgeführt. So kann bspw. festgelegt werden welche Reaktion auf das Entdecken eines anderen Bluetooth-Geräts folgt oder wie mit über L2CAP empfangenen Anwendungsdaten verfahren wird.

\textit{mbedTLS} ist eine Software-Bibliothek für TLS und eignet sich aufgrund niedriger Anforderungen an den Speicher für eingebettete Systeme. Aktuell wird höchstens TLS 1.2 unterstützt \cite{ArmMbedCoreFeatures}.

\subsubsection{Smartphone/Android}
Das Smartphone unterstützt Bluetooth 5.0 und wird mit Android 11.0 betrieben. Wenn es mit dem \textit{ESP32-WROOM-32} per Bluetooth kommuniziert, wird also die Bluetooth-Version 4.2 verwendet. Android unterstützt mit den Programmpaketen "`android.bluetooth.le"' \cite{android_le} BLE und mit "`javax.net.ssl"' \cite{android_ssl} SSL/TLS. Wie in Sektion \ref{sec: infra sicherheit} erläutert, sollten nur TLS 1.2 oder TLS 1.3 genutzt werden. TLS 1.2 wird ab Android 4.1 und TLS 1.3 ab Android 10.0 unterstützt \cite{android_ssl_context}.

		\subsubsection{Simulierung der Zertifizierungsstelle}
			\label{sec: impl prototyp zertifizierungstelle}
			Die Zertifizierungstelle wird mittels eines Personal Computer (PC) simuliert. Die Zertifikate und Schlüssel werden vom PC mittels \textit{mbedTLS} generiert und beim "`Flashen"' des jeweiligen Programmes auf die Mikrocontroller als Dateien übertragen. Jedoch werden sie nicht in Keystores gespeichert. Die Zertifkate werden nach dem Standard X.509 erstellt und die Schlüssel mittels des RSA-Verfahrens generiert. In Abb. \ref{fig: impl certs keys} wird gezeigt, wie sie erstellt bzw. generiert werden.
\begin{figure}[H]
    \centering
    \includegraphics[width=0.9\textwidth]{graphics/impl_keys_certs.pdf}
    \caption[Generierung der Schlüssel und Erstellung der Zertifikate]{Generierung der Schlüssel und Erstellung der Zertifikate}
    \label{fig: impl certs keys}
\end{figure}
Zuerst wird ein 4096 Bit langer privater Schlüssel für die Zertifizierungsstelle generiert. Danach wird das zugehörige Root-Zertifikat mit diesem Schlüssel erstellt. Dabei wird angegeben, dass es ein selbstunterzeichnetes Zertifikat ist und für eine Zertifizierungsstelle ausgestellt wird. Der Name des Ausstellers ist in diesem Fall auch der Name des Eigentümers. Alle weiteren privaten Schlüssel sind ebenfalls 4096 Bit lang. Wurden sie generiert, wird mit jedem privaten Schlüssel genau ein Zertifikat erstellt, dessen Ausstellerzertifikat das Root-Zertifikat ist.
\\\\
Nun liegen alle Zertifikate und Schlüssel vor und werden beim "`Flashen"' des jeweiligen Programmes folgendermaßen auf die beiden Mikrocontroller übertragen:
\begin{itemize}
    \item Server-Mikrocontroller (Fahrzeug):
    \begin{itemize}
        \item privater Server-Schlüssel (Fahrzeug)
        \item Server-Zertifikat (Fahrzeug)
        \item Root-Zertifikat
    \end{itemize}
    \item Client-Mikrocontroller (stellvertretend für die App):
    \begin{itemize}
        \item privater Client-Schlüssel (App)
        \item Client-Zertifikat (App)
        \item privater Subscription-Schlüssel
        \item Subscription-Zertifikat
        \item Root-Zertifikat
    \end{itemize}
\end{itemize}
Der Client erhält zusätzlich den Schlüssel und das Zertifikat zum Erstellen der Subscriptions, um damit das Back End zu simulieren (siehe Sektion \ref{sec: impl prototyp back end}).

		\subsubsection{Ablauf der Anwendung für Client und Server}
			\label{sec: impl prototyp anwendung}
			Entsprechend Abb. \ref{fig: impl ablauf anwendung teil 1} ist der erste Teil des Ablaufs der Anwendungen für Client-Mikrocontroller bzw. Server-Mikrocontroller dargestellt. Er ist für beide Anwendungen identisch.
\begin{figure}[H]
    \centering
    \includegraphics[width=1\textwidth]{graphics/ablauf_anwendung_teil_1.pdf}
    \caption[Ablauf beider Anwendungen (Teil 1)]{Ablauf beider Anwendungen (Teil 1)}
    \label{fig: impl ablauf anwendung teil 1}
\end{figure}
Zu Beginn wird das Dateisystem SPIFFS konfiguriert und initialisiert. Vor dem "`Flashen"' wird eine Partition erstellt, die die jeweiligen Schlüssel und Zertifikate enthält (siehe Sektion \ref{sec: impl prototyp zertifizierungstelle}). Danach wird über \textit{nimBLE} der Bluetooth-Controller im Modus Low Energy und das Host Controller Interface initialisert. Danach wird Speicher für zwei Memory Pools reserviert: einen Memory Pool für empfangene Anwendungsdaten und einen für zu sendende Anwendungsdaten. Schließlich wird der nimBLE-Host in einem neuen Task initialisiert, um das Event Handling für L2CAP und GAP separiert von der Anwendung zu betreiben.
\\\\
Die Abb. \ref{fig: impl ablauf anwendung server teil 2} zeigt den weiteren Ablauf der Server-Anwendung.
\begin{figure}[H]
    \centering
    \includegraphics[width=0.8\textwidth]{graphics/ablauf_anwendung_teil_2_server.pdf}
    \caption[Ablauf der Server-Anwendung (Teil 2)]{Ablauf der Server-Anwendung (Teil 2)}
    \label{fig: impl ablauf anwendung server teil 2}
\end{figure}
Zunächst wird der L2CAP-Server erstellt. Somit wird das L2CAP Event Handling aktiv. Anschließend wird der TLS-Kontext erstellt. Er benötigt den privaten Schlüssel und das Zertifikat des Fahrzeugs sowie das Root-Zertifikat. Zudem verlangt er nach den Funktionen für das Senden bzw. Empfangen von Daten durch den Transport (L2CAP). Danach beginnt die Server-Anwendung mit dem Advertising mithilfe von GAP. Dabei werden Connectable Undirected Advertising Events als Broadcast gesendet, in denen die öffentliche Bluetooth-Adresse des Server-Mikrocontrollers angegeben wird. Nun ist auch das GAP Event Handling aktiv.
\\\\
In Abb. \ref{fig: impl ablauf anwendung client teil 2} ist der weitere Ablauf der Client-Anwendung dargestellt.
\begin{figure}[H]
    \centering
    \includegraphics[width=0.6\textwidth]{graphics/ablauf_anwendung_teil_2_client.pdf}
    \caption[Ablauf der Client-Anwendung (Teil 2)]{Ablauf der Client-Anwendung (Teil 2)}
    \label{fig: impl ablauf anwendung client teil 2}
\end{figure}
Die Client-Anwendung erstellt ebenfalls einen TLS-Kontext, nur dass hier der private Schlüssel und das Zertifikat für den Client und nicht für das Fahrzeug benötigt wird. Danach wird mittels GAP nach Advertising Events gescannt (General Discovery Mode). Somit wird das GAP Event Handling aktiv.
\\\\
Mittels der beiden Event Handlings bauen Server und Client erst eine Verbindung über GAP auf und anschließend eine Verbindung über einen neuen L2CAP-Channel. Daraufhin wird der TLS-Handshake ausgeführt. Dabei senden die Anwendungen über L2CAP Service Data Units, deren maximale Länge der Maximum Transmission Unit (MTU) entspricht. Empfängt eine Anwendung eine SDU, speichert sie die SDU im Memory Pool für empfangene Anwendungsdaten (RX-Buffer). Der RX-Buffer ist in Blöcke gleicher Länge aufgeteilt, die jeweils eine MTU speichern können. Jede im RX-Buffer gespeicherte SDU nimmt einen dieser Blöcke ein. Möchte nun die Anwendung eine bestimmte Anzahl an Bytes empfangen, prüft sie zunächst, ob sich eine oder mehrere SDUs im RX-Buffer befinden. Sie liest ggf. die geforderte Byteanzahl aus der SDU und lässt einen Pointer auf das nächst zu lesende Byte zeigen. Für den Fall eines leeren RX-Buffers, wartet die Anwendung (mittels Semaphores) bis eine SDU empfangen und in den RX-Buffer geschrieben wurde.
\\\\
Schlägt der TLS-Handshake fehl, stoppen beide Anwendungen. Anderenfalls erstellt der Client die Subscription mit einem vorgefertigtem Payload. Hat er den Hashwert des Payloads gebildet und signiert, sendet er die Subscription mithilfe des TLS-Kontextes an den Server. Dieser verifiziert die Subscription (siehe Sektion \ref{sec: impl verbindungsaufbau und autorisierung}). Wurde die Subscription erfolgreich verifiziert, erfolgt die Bestätigung auf der Konsole. Anderenfalls stoppt die jeweilige Anwendung.

		\subsubsection{Simulierung des Back Ends}
			\label{sec: impl prototyp back end}
			\input{sections/implementierung/prototyp/simulierung_des_backends.tex}

\newpage
\section{Ausblick}
	\input{sections/zusammenfassung_ausblick/ausblick.tex}

\newpage
\section{Zusammenfassung}
	Zu Beginn der Arbeit wurde die Funktionsweise von \textit{Bluetooth Low Energy} (BLE) vorgestellt. Die Recherche beschränkte sich auf die Spezifikationen und Webauftritte der Bluetooth SIG, um aus erster Hand ein Verständnis für die BLE-Architektur zu gewinnen. Durch die Untersuchung der Sicherheitsfunktionen und weitere Recherche, konnten die Schwachstellen von BLE aufgedeckt werden.
\\\\
Zum Entwurf der Infrastruktur musste zunächst ein geeigneter Transport innerhalb der BLE-Architektur bestimmt werden. Die Tatsache, dass sich viele Quellen zu dieser Frage auf das Protokoll GATT beziehen, war irreführend. GATT ist aufgrund seiner Attributsemantik und dem einhergehenden Overhead in der Übertragung ungeeignet für den effizienten Transport von Daten. Stattdessen stellt sich L2CAP als geeignetes Transportprotokoll heraus. Desweiteren war eine Recherche bzgl. der Sicherheitprobleme der Kommunikation und deren Lösungen notwendig. Das TLS-Protokoll stellte sich dabei als hervorragende Lösung heraus, um über einen beliebigen Transport sicher Daten zu übertragen.
\\\\
Die Implementierung der Infrastruktur für den Verleihdienst wies das Problem der Autorisierung auf. Als Lösung wurde die "`Subscription"' entworfen, die sicherstellt, dass ein Nutzer berechtigt ist, ein Fahrzeug auszuleihen, und es nach kurzer Abwesenheit wieder zu nutzen. Der entwickelte Prototyp für die Implementierung zeigt, wie für zwei Mikrocontroller die Infrastruktur angewandt wird und eine Subscription sicher übertragen und verifiziert wird.
\\\\
Eine Weiterführung der Arbeit wäre in der Implementierung denkbar. Dabei könnte der Prototyp zunächst unter Einbezug eines Smartphones weiterentwickelt werden. Desweiteren könnten die Zertifizierungsstelle und der Back End Server physisch eingebunden werden, um einen realistischen Prototypen zu testen.

Außerdem kann die Arbeit mit einer genauen Kryptoanalyse des vorgestellten Konzepts der "`Subscription"' weitergeführt werden, um eventuelle Schwachstellen aufzudecken und auszuschließen.

\newpage
\printbibliography[heading=bibintoc]

\newpage
\appendix

\section{Beispiele}
	\subsection{Sequence Number / Next Expected Sequence Number}
		\label{sec: anhang nesn sn}
		\begin{figure}[H]
    \centering
    \includegraphics[width=0.9\textwidth]{graphics/nesn_sn_bsp.pdf}
    \caption[Beispiel für SN/NESN]{Beispiel für SN/NESN}
    \label{fig: anhang nesn sn}
\end{figure}

Gerät A und B verfügen jeweils über eigene Werte für SN/NESN und setzten diese jeweils auf 0 zu Beginn der Verbindung. Nun senden A und B gegenseitig Data Channel PDUs und nutzen folgenden Algorithmus für die SN/NESN. Die SN/NESN einer Data Channel PDU werden zur SN Bit bzw. NESN Bit genannt. Bei dem erstmaligen Empfang von Data Channel PDU \#4 gibt der CRC einen Fehler aus, also wurde das Paket nicht korrekt übertragen. 

Algorithmus zum Setzen der SN/NESN \cite{BtSpec4.0_2239-2241} (bezieht sich immer nur auf das ausführende Gerät):
\\\\
Sei "`neue Data Channel PDU"' eine Data Channel PDU, die erstmalig gesendet wird (also keine erneute Sendung einer bereits gesendeten Data Channel PDU).\\
Sei "`alte Data Channel PDU"' eine Data Channel PDU, die bereits von dem Gerät gesendet bzw. empfangen wurde.\\
Sei "`letzte Data Channel PDU"' eine Data Channel PDU, die das Gerät zuletzt gesendet hat.

\begin{itemize}
    \item Senden einer Data Channel PDU:
    \begin{enumerate}
        \item Setze NESN Bit auf NESN. Falls Data Channel PDU eine neue Data Channel PDU ist, setze SN Bit auf SN
    \end{enumerate}
    \item Empfangen einer Data Channel PDU:
    \begin{enumerate}
        \item Ist SN Bit gleich NESN, dann wurde eine neue Data Channel PDU empfangen und NESN wird inkrementiert. Anderenfalls ist SN Bit ungleich NESN, also wurde eine alte Data Channel PDU empfangen, die ignoriert wird.
        \item Ist NESN Bit ungleich SN, dann wurde die letzte Data Channel PDU vom gegenüber bestätigt (ACK) und es wird SN inkrementiert. Anderenfalls ist NESN Bit gleich SN, also wurde die letzte Data Channel PDU nicht bestätigt (NAK) und muss erneut gesendet werden. Dabei muss SN Bit den Wert des SN Bit der letzten Data Channel PDU (also die zuletzt vom Gerät gesendet wurde) annehmen.
    \end{enumerate}
\end{itemize}

% - senden neuer Daten: SN bit auf SN setzen
% - senden Daten: NESN bit auf NESN setzen

% - empfangen Daten:
% SN bit ==  NESN, dann wurden neue Daten empfangen, NESN ++
% SN bit != NESN, dann wurden alte Daten empfangen
% NESN bit != SN, dann zuletzt gesendete Daten wurden ack., SN++
% NESN bit == SN, dann zuletzt gesendete Daten nak, diese erneut senden
% und dabei SN bit auf den Wert der Data Channel PDU
% setzen, die nicht empfangen wurde

\section{Sequenzdiagramme}
	\subsection{Verbindungsaufbau der Infrastruktur}
		\label{sec: anhang infra verb aufbau}
		\begin{figure}[H]
    \centering
    \includegraphics[width=1.05\textwidth]{graphics/infra_verb_aufbau.pdf}
    \caption[Verbindungsaufbau der Infrastruktur]{Verbindungsaufbau der Infrastruktur}
    \label{fig: infra verb aufbau}
\end{figure}

\end{document}