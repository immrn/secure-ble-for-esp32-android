Der Security Manager bzw. das Security Manager Protocol (SMP) ist dafür zuständig eine sichere Verbindung zwischen zwei Geräten (Master und Slave) aufzubauen. Dies umfasst im Wesentlichen das Pairing, welches zur Authentifizierung und Generierung eines Schlüssels dient, und die Verteilung von Schlüsseln. Entsprechend der Abbildung X lässt sich das SMP in die BLE-Architektur einordnen.
% TODO BILD VERWEIS
% TODO BILD Spec 4.0 S. 1958 Figure 1.1
% TODO QUELLE Spec 4.0 S. 1958

Das Pairing lässt sich in drei Phasen aufteilen. Phase 1 und 2 dienen der Generierung eines Schlüssels, den beide Geräte zur Verschlüsselung und Authentifizierung im Link Layer nutzen. Dafür wird der Cipher Block Chaining - Message Authentication Code (CCM) Mode in Verbindung mit dem Advanced Encryption Standard (AES), genauer dem AES-128 Block Cipher, genutzt.\\\\
% TODO QUELLE Spec. 4.0 S. 2285, 1 ENCRYPTION AND AUTHENTICATION OVERVIEW

\paragraph{Pairing: Phase 1} \mbox{} \vspace{0.2cm} \\

Die erste Phase ist der Pairing Feature Exchange, bei dem die beiden Geräte ihre für den Nutzer zugänglichen Ein- und Ausgabemöglichkeiten (IO Capabilities) austauschen. Die Tabellen X und Y 
% TODO TABELLE VERWEIS
listen die entsprechenden Bezeichnungen dieser Möglichkeiten auf.

\begin{table}
    \begin{tabularx}{\textwidth}{|l|X|}
    \hline
    \textbf{Eingabemöglichkeit} & \textbf{Beschreibung} \\
    \hline
    Keine Eingabe & Gerät kann keine Eingaben entgegennehmen \\
    \hline
    Ja / Nein & Gerät kann zwei verschiedene Eingaben verarbeiten, die der Bedeutung Ja bzw. Nein zugewiesen werden können (bspw. zwei Tasten) \\
    \hline
    Tastatur & Eingabe der Ziffern 0 bis 9 möglich und die Möglichkeit Ja und Nein einzugeben \\
    \hline
    \end{tabularx}
    \caption{Eingabemöglichkeiten eines Gerätes}
\end{table}

\begin{table}
    \begin{tabularx}{\textwidth}{|l|X|}
    \hline
    \textbf{Ausgabemöglichkeiten} & \textbf{Beschreibung} \\
    \hline
    Keine Ausgabe & Gerät kann keine 6-stellige Dezimalzahl dem Nutzer anzeigen bzw. kommunizieren \\
    \hline
    Numerische Ausgabe & Gerät kann eine 6-stellige Dezimalzahl dem Nutzer anzeigen bzw. kommunizieren \\
    \hline
    \end{tabularx}
    \caption{Ausgabemöglichkeiten eines Gerätes}
\end{table}

% TODO QUELLE für Tabellen: Spec 4.0 S. 1965 Table 2.1 und Table 2.2

Desweiteren tauschen die Geräte in der ersten Phase Informationen darüber aus, ob eine Authentifizierung zum Schutz vor einem Man-In-The-Middle-Angriff nötig ist (MITM Flag), und ob Daten für die Authentifizierung über die Pairing-Methode Out Of Band (OOB), d.h. mittels einer anderen Technologie (z.B. Near Field Communication), übertragen werden können (OOB Flag). Außerdem werden die minimalen und maximalen Größen für die Schlüssel ausgetauscht, die sich in 8 Bit großen Schritten zwischen 56 Bit (7 Byte) und 128 Bit (16 Byte) befinden. Dabei wird der kleinere Wert beider maximaler Größen übernommen. Falls die beiden Spannen sich nicht schneiden, wird das Pairing abgebrochen.


\paragraph{Pairing: Phase 2} \mbox{} \vspace{0.2cm} \\

Anhand der ausgetauschten Informationen aus der ersten Phase wird in der zweiten Phase entschieden, welche der folgenden Methoden zur Generierung des Short Term Key (STK) bzw. Long Term Key (LTK) zu verwenden ist:

\begin{itemize}
    \item{Numeric Comparison (für LE erst ab Bluetooth-Version 4.2)}
    \item{Just Works}
    \item{Out of Band (OOB)}
    \item{Passkey Entry}
\end{itemize}

Da das Pairing (speziell Phase 2) für LE in der Bluetooth-Version 4.0 funktionale Unterschiede zum Pairing für LE ab Version 4.2 aufweist, wird das Pairing für LE in Version 4.0 als \textbf{LE Legacy Pairing} und das Pairing für LE ab Version 4.2 als \textbf{LE Secure Connections Pairing} bezeichnet. Aus Sicht des Nutzers sind diese jedoch gleich und jede Bluetooth-Version, die LE unterstützt, unterstützt das LE Legacy Pairing. Im Gegensatz zum LE Secure Connections Pairing bietet LE Legacy Pairing für die Methoden Just Works und Passkey Entry keinen Schutz vor passivem Abhören während des Pairings, da LE Secure Connections Elliptic Curve Diffie-Hellman (ECDH) für den Schlüsselaustausch nutzt und LE Legacy Pairing nicht.\\\\
% TODO QUELLE Spec 4.2 S. 248 5.4.1 Association Models

Haben beide Geräte OOB-Authentifizierungsdaten für das LE Legacy Pairing, wird unabhängig von der jeweiligen MITM-Flag die Methode OOB gewählt. In LE Secure Connections Pairing wird ebenso verfahren, nur dass hier nicht die OOB-Flag beider Geräte gesetzt sein müssen (aber können), da eine gesetzte OOB-Flag genügt. Ist die MITM-Flag beider Geräte nicht gesetzt, wird die Methode Just Works ausgeführt. Anderenfalls werden die Ein- und Ausgabemöglichkeiten der Geräte für die Wahl der Methode einbezogen (siehe Anhang Tabelle X).\\\\
% TODO ANHANG TABELLE VERWEIS
% TODO ANHANG TABELLE Spec 4.2 S. 2302, Table 2.8 (Achtung: Seitenumbruch)



\subparagraph{Pairing Methoden} \mbox{} \vspace{0.2cm} \\

Bei der \textbf{Numeric Comparison} wird dem Nutzer auf beiden Geräten jeweils eine zufällig generierte sechsstellige Dezimalzahl angezeigt. Diese muss der Nutzer vergleichen und im Falle der Übereinstimmung auf beiden Geräten bestätigen oder anderenfalls ablehnen. Somit kann der Nutzer unabhängig von der Namensgegebung der Geräte sicherstellen, die richtigen Geräte ausgewählt zu haben. Zudem bietet diese Methode Schutz vor MITM-Angriffen. Außenstehende, die Kenntnis über diese Zahl gewonnen haben, können laut X
% TODO QUELLE VERWEIS Sepc 4.2 S. 244 5.2.4.1 Numeric Comparison
damit keinen Vorteil zur Entschlüsselung der zwischen den beiden Geräten ausgetauschten Daten erlangen, da die Zahl nicht als Eingabe zur Generierung eines Schlüssels verwendet wird.

\textbf{Just Works} basiert auf der Funktionsweise von Numeric Comparison mit dem Unterschied, dass hier dem Nutzer keine sechsstellige Dezimalzahl ausgegeben wird und er die Verbindung nur bestätigen muss. Dadurch bietet Just Works keinen Schutz vor MITM-Angriffen, aber einen Schutz gegen passives Abhören (außer für LE Legacy Pairing).
% TODO QUELLE Spec 4.2 S.245 5.2.4.2 Just Works

\textbf{Out of Band} ist die Nutzung einer anderen Technologie (z.B. Near Field Communication), um Geräte zu entdecken oder um kryptographische Informationen für den Pairing-Prozess auszutauschen. Dabei sollte die Technologie Schutz vor MITM-Angriffen bieten.
% TODO QUELLE Spec 4.2 S. 246 5.2.4.3 Out of Band

Die Methode \textbf{Passkey Entry} definiert, dass ein Gerät die zufällig generierte sechstellige Dezimalzahl ausgibt und der Nutzer diese auf dem anderen Gerät eingeben muss. 
% TODO QUELLE Spec 4.2 S. 246 5.2.4.4 Passkey Entry
Ein Schutz gegen MITM-Angriffe existiert, da diese nur mit einer Wahrscheinlichtkeit von 0,000001 für jede Durchführung der Methode möglich sind. Schutz gegen passives Abhören bietet Passkey Entry nur in LE Secure Connections Pairing und nicht in LE Legacy Pairing.
% TODO QUELLE Spec. 4.2 S. 2304 2.3.5.3 LE Legacy Pairing - Passkey Entry



\subparagraph{LE Legacy Pairing: Schlüssel und deren Generierung} \mbox{} \vspace{0.2cm} \\

Beim LE Legacy Pairing zweier Geräte generieren beide einen 128 Bit langen Temporary Key (TK), der bei der Authentifizierung genutzt wird, um den STK zu generieren und die Verbindung zu verschlüsseln. In Just Works wird der TK auf null gesetzt. Bei der Methode Passkey Entry ist der TK die besagte zufällig generierte sechstellige Dezimalzahl, die bereits mit 20 Bit dargestellt werden kann, weswegen die restlichen Bit des TK auf null gesetzt werden müssen. Dagegen kann bei OOB auf diese Einschränkung verzichtet werden, wodurch der TK wahrhaftig eine Länge von 128 Bit besitzt.

Das Gerät, welches das Pairing einleitet (Master), generiert eine zufällige 128 Bit große Nummer \textit{Mrand} und ermittelt den 128 Bit großen Bestätigungswert \textit{Mconfirm} mit der Confirm Value Generation Function c1 [X]. 
% TODO QUELLE verweis auf Spec. 4.2 S. 2288 2.2.3 Confirm value generation function c1 for LE Legacy Pairing
Für diese sind die Eingabewerte \textit{TK}, \textit{Mrand}, der Pairing Request Command, der Pairing Response Command, der Adresstyp des Masters, die Adresse des Masters, der Adresstyp des Slaves und dessen Adresse. Ebenso führt das antwortende Gerät diese Schritte durch, wobei \textit{Mrand} als \textit{Srand} bezeichnet wird und \textit{Mconfirm} als \textit{Sconfirm}. Die Eingabewerte der Funktion c1 bleiben die gleichen, nur dass \textit{Mrand} durch \textit{Srand} ersetzt wird. Danach findet entsprechend Abb. X folgender Austausch statt.
% TODO BILD VERWEIS
\begin{figure}[hbt!]
    \centering
    % TODO \includegraphics[width=0.5\linewidth]{graphics/LE_Legacy_Pairing_STK_Generation.svg}
    \caption{STK Generierung in LE Legacy Pairing}
\end{figure}

Anschließend wird der STK mit der Funktion s1 [X] 
% TODO QUELLE verweis auf Spec 4.2 S. 2290 2.2.4 Key generation function s1 for LE Legacy Pairing
und deren Eingabewerte \textit{TK}, \textit{Srand} und \textit{Mrand} generiert. Demnach kann bei der Methode Passkey Entry kein ausreichender Schutz gegen passives Abhören geboten werden, da der TK nur wenig mögliche Werte annehmen kann. Ist die vereinbarte Schlüsselgröße kleiner als 128 Bit, werden die überschüssigen Bit beginnend bei dem Bit mit dem höchsten Stellenwert auf null gesetzt. Der STK wird nun zur Verschlüsselung der Verbindung genutzt.
% TODO QUELLE Spec. 4.2 S. 2305 f. 2.3.5.5 LE Legacy Pairing Phase 2



\subparagraph{LE Secure Connections Pairing: Schlüssel und deren Generierung} \mbox{} \vspace{0.2cm} \\

Beim LE Secure Connections Pairing wird ein Long Term Key (LTK) erstellt. Zuvor erstellen beide Geräte jeweils ein ECDH-Schlüsselpaar (PK - Public Key, SK - Private Key) und tauschen ihre Public Keys aus. Danach berechnet jedes Gerät den Diffie-Hellman-Schlüssel aus seinem Private Key und dem Public Key des Anderen. Durch den Diffie-Hellman-Schlüssel kennen beide Parteien ein gemeinsames Geheimnis, mit dem sie den weiteren Datenaustausch zur Authentifizierung verschlüsseln können. Die Authentifizierung ist notwendig, da der ECDH-Schlüsselaustausch zwar resistent gegen passives Abhören ist, jedoch nicht gegen MITM-Angriffe.
% TODO QUELLE Spec. 4.2 S. 2307 2.3.5.6.1 Public Key Exchange

Diese Authentifizierung wird mit den Pairing Methoden Numeric Comparison, Just Works, OOB und Passkey Entry ermöglicht. Jedoch unterscheiden diese sich aus funktionaler Sicht (nicht aus Nutzersicht) zum LE Legacy Pairing durch komplexere Verfahren. Letztendlich lässt sich für die vier Pairing-Methoden Folgendes zusammenfassen. Numeric Comparison signalisiert dem Nutzer mit einer Wahrscheinlichkeit von 0,999999 einen stattfindenden MITM-Angriff. 
% TODO QUELLE Spec. 4.2 S. 2309, 2.3.5.6.2 Authentication Stage 1 – Just Works or Numeric Comparison
Just Works bietet keinen Schutz vor einem MITM-Angriff. 
% TODO QUELLE Spec. 4.2 S. 245, 5.2.4.2 Just Works
Ein MITM-Angriff während des Passkey Entry gelingt nur mit einer Wahrscheinlichkeit von 0,000001. 
% TODO QUELLE Spec. 4.2 S. 2311, 2.3.5.6.3 Authentication Stage 1 – Passkey Entry
Wie anfällig OOB für Angriffe ist hängt von der verwendeten OOB-Technologie ab.
% TODO QUELLE Spec. 4.2 S. 2312, 2.3.5.6.4 Authentication Stage 1 – Out of Band

Nach der Ausführung einer Pairing-Methode wird der LTK als Teilergebnis der Funktion f5 [X] 
% TODO QUELLE Spec. 4.2 S. 2292, 2.2.7 LE Secure Connections Key Generation Function f5
mit den Eingabewerten Diffie-Hellman-Key, ein Nonce des Masters, ein Nonce des Slaves und der Adresse des Masters und Slaves ermittelt.\\\\
% TODO QUELLE Spec. 4.2 S. 2314, 2.3.5.6.5 Authentication Stage 2 and Long Term Key Calculation

\paragraph{Pairing: Phase 3} \mbox{} \vspace{0.2cm} \\

Wurde der STK bzw. LTK generiert, wird dieser genutzt, um die Verbindung zu verschlüsseln. Nun können in der dritten Phase transportspezifische Schlüssel ausgetauscht werden. Z.B. wird der Identity Resolving Key (IRK) zur Generierung und Auflösung von zufälligen Adressen verwendet und der Connection Signature Resolving Key (CSRK) zur Signatur von Daten und Überprüfung von Signaturen.

% Für Numeric Comparison und Just Works wird entsprechend Abb. X fortgefahren.
% % TODO BILD VERWEIS

% \begin{figure}[hbt!]
%     \centering
%     \inlcudegraphics[width=0.5\linewidth]{graphics/LE_Secure_Connections_Pairing_Numeric_Comparison_Just_Works.svg}
%     \caption{}
% \end{figure}

% Der zu Beginn zufällig generierte Nonce (N\textunderscore a bzw. N\textunderscore b) mit einer Länge von 128 Bit wird für jeden Durchlauf neu erzeugt und schützt vor Replay-Angriffen. Für die Berechnung der Bestätigung C\textunderscore b wird die Einwegfunktion f4 [X] genutzt. 
% % TODO QUELLE Spec. 4.2 S. 2291 2.2.6 LE Secure Connections Confirm Value Generation Function f4
% Mittels der Funktion g2 [X] 
% % TODO QUELLE Spec 4.2 S. 2295 2.2.9 LE Secure Connections Numeric Comparison Value Generation Function g2
% werden die Dezimalzahlen D\textunderscore a und D\textunderscore b berechnet. Darauf werden diese dem Nutzer auf den Geräten ausgegeben und dieser muss deren Gleichheit bestätigen. Wird anstatt Numeric Comparison die Methode Just Works ausgeführt, dann werden D\textunderscore a und D\textunderscore b nicht berechnet und folglich nicht dem Nutzer gezeigt. Sollte ein Fehler auftreten wird das Protokoll abgebrochen und kann neu gestartet werden.

% Beim vorherigen ECDH-Schlüsselaustausch, kann ein MITM-Angriff angewandt werden. Eine einfache Variante, die keine Informationen der beiden angegriffenen Geräte zwischen diesen austauscht sondern mit diesen jeweils separat die Methode Numeric Comparison durchführt, endet darin, dass dem angegriffenem Nutzer mit einer Wahrscheinlichkeit von 0,999999 auf dessen Geräten zwei verschiedene Dezimalzahlen angezeigt werden. Eine andere Variante wäre aus Sicht des Angreifenden den Verkehr bestehend aus der Bestätigung C\textunderscore b, dem Nonce N\textunderscore a und N\textunderscore b nur weiterzuleiten. Jedoch wird der Master bei der Überprüfung der Bestätigung C\textunderscore b feststellen, dass diese nicht mit seinem Public Key PK\textunderscore a erstellt wurde, was zum Abbruch führt.\\\\
% % TODO QUELLE Spec. 4.2 S. 2308 f. 2.3.5.6.2 Authentication Stage 1 – Just Works or Numeric Comparison
% Passkey Entry
% OOB