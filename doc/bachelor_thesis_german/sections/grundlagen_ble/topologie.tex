Ein Ad-Hoc-Netzwerk bestehend aus Bluetooth-Geräten wird Piconet genannt. Dabei existiert in einem Piconet immer ein Master-Gerät, das sich mit einem oder mehreren Slave-Geräten verbinden kann. Ein Gerät kann zur selben Zeit innerhalb mehrerer Piconets agieren, wobei die Rollen unabhängig sind. Z.B. könnte ein Gerät jeweils der Master eines Piconet A und eines Piconet B sein, während es noch ein Slave in einem Piconet C ist.
% TODO QUELLE Specification 4.0 PDF S. 181

Um mittels BLE ein Piconet zu bilden, benötigt es einen Advertiser. Auf vorgegebenen Frequenzen sendet dieser unadressiert Daten, mit denen er sich für andere Geräte bemerkbar macht (Advertisements). Dabei können Advertisements auch genutzt werden, um Nutzungsdaten zu senden. Geräte, die diese Frequenzen nach Advertisement-Paketen scannen, werden Scanner bzw. Initiator genannt. Auf diesem Weg finden sich die Geräte (Discovering) und können sich verbinden. Der Initiator unterscheidet sich vom Scanner, da er in der Lage ist, sich zu einem Advertiser zu verbinden, von dem er ein Advertisement erhielt, dass das Verbinden zu diesem ermöglicht.