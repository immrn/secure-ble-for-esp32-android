Um miteinander zu kommunizieren, müssen zwei Bluetooth"=Geräte (ein Sender und ein Empfänger) zur selben Zeit den selben Kanal nutzen, wobei sich der Empfänger in der Reichweite des Senders befinden muss. Da mehrere Piconets zur selben Zeit im selben lokalen Bereich agieren können, besteht die Wahrscheinlichkeit, dass zwei Sender zweier verschiedener Gerätepaare in Reichweite den selben Kanal zur selben Zeit nutzen und eine Kollision verursachen.
\\\\
Mittles des Frequenzmultiplexverfahrens ist das ISM"=Band eines LE"=Systems von 2400 MHz bis 2483,5 MHz in 40 Funkkanäle unterteilt. Beginnend bei 2402 MHz nutzt jeder Kanal eine Frequenz, die 2 MHz über der Frequenz des Vorgängers liegt. Das Trägersignal wird mithilfe des Gaussian Frequency Shift Keying moduliert. \cite{BtSpec4.0_2180-2181}
% QUELLE Specification 4.0 PDF S. 2180-2181
\\\\
Somit bildet der Physical Channel die niedrigste Ebene der Architektur. 37 der 40 Kanäle werden als LE Piconet Channel (entsprechend S. \pageref{fig: controller architektur} Abb. \ref{fig: controller architektur} als LE Piconet Physical Channel) bezeichnet, die mit einem Piconet assoziiert werden und zur Kommunikation zwischen zwei bereits verbundenen Geräten dienen. Die verbleibenden drei Kanäle werden Advertisement Broadcast Channel (entsprechend S. \pageref{fig: controller architektur} Abb. \ref{fig: controller architektur} als LE Advertising Physical Channel) genannt und befinden sich auf den Frequenzen 2402 MHz, 2426 MHz sowie 2480 MHz. \cite{BtSpec4.0_2199}
% QUELLE advertisment frequenzen Specification 4.0 PDF S. 2199
\\\\
Mittels Advertisements können Geräte in diesen drei Kanälen auf sich aufmerksam machen, um von anderen Geräten entdeckt zu werden. Zudem werden sie genutzt, um Geräte miteinander zu verbinden oder Anwendungsdaten an Scanner bzw. Initiatoren zu senden. Ein Gerät kann nur einen Kanal zur selben Zeit nutzen, weswegen das Zeitmultiplexverfahren verwendet wird, das bereits verbundenen Geräten ermöglicht, zusätzlich das Advertisement zu betreiben.
\\\\
Um Interferenzen innerhalb des genutzten Frequenzbands z.B. mit \textit{Wi-Fi} zu vermeiden, wird das Adaptive Frequency Hopping \cite{BtAfh} genutzt (eine Form des Frequency Hopping Spread Spectrum). Dabei wechseln Sender und Empfänger in kurzen Zeitabständen den Kanal und passen die Menge der zu nutzenden Kanäle (Channel Map) an, indem sie dynmaisch ermittleln, in welchen Kanälen häufiger Kollisionen auftreten. Treten in einem Kanal häufig Interferenzen auf, wird er für eine bestimmte Zeitspanne aus der Channel Map entfernt und vorerst nicht mehr genutzt.
% QUELLE https://www.bluetooth.com/blog/how-bluetooth-technology-uses-adaptive-frequency-hopping-to-overcome-packet-interference/