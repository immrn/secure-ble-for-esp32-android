Um miteinander zu kommunizieren, müssen zwei Bluetooth-Geräte (ein Sender und ein Empfänger) zur selben Zeit den selben Kanal nutzen, wobei sich der Empfänger in der Reichweite des Senders befinden muss. Da mehrere Piconets zu selben Zeit im selben lokalen Bereich agieren können, besteht die Wahrscheinlichkeit, dass zwei Sender zweier verschiedener Gerätepaare in Reichweite den selben Kanal zur selben Zeit nutzen, wodurch eine Kollision resultiert.
\\\\
Mittles des Frequenzmultiplexverfahrens ist das ISM-Band eines LE-Systems von 2400 MHz bis 2483,5 MHz in 40 Funkkanäle aufgeteilt. Beginnend bei 2402 MHz nutzt jeder Kanal eine Frequenz, die 2 MHz über der Frequenz des Vorgängers liegt.
% TODO QUELLE Specification 4.0 PDF S. 2180
Das Trägersignal wird mithilfe des Gaussian Frequency Shift Keying moduliert.
% TODO QUELLE Specification 4.0 PDF S. 2181
Somit bildet der Physical Channel die niedrigste Ebene der Architektur. 37 der 40 Kanäle werden als LE Piconet Channel bezeichnet, die mit einem Piconet assoziiert werden und zur Kommunikation zwischen zwei bereits verbundenen Geräten dienen. Die verbleibenden drei Kanäle werden Advertisement Broadcast Channel genannt und befinden sich auf den Frequenzen 2402 MHz, 2426 MHz sowie 2480 MHz.
% TODO QUELLE advertisment frequenzen Specification 4.0 PDF S. 2199

Mittels Advertisements können Geräte in diesen drei Kanälen auf sich aufmerksam machen, damit andere Geräte diese entdecken können. Zudem werden diese genutzt, um Geräte miteinander zu verbinden oder Anwenderdaten an Scanner bzw. Initiatoren zu senden. Ein Gerät kann nur einen Kanal zur selben Zeit nutzen, weswegen das Zeitmultiplexverfahren verwendet wird, welches bereits verbundenen Geräten ermöglicht, zusätzlich das Advertisement zu betreiben.
\\\\
Um Interferenzen innerhalb des genutzten Frequenzbands z.B. mit Wi-Fi zu vermeiden, wird das Adaptive Frequency Hopping genutzt (eine Form des Frequency Hopping Spread Spectrum). Dabei wechseln Sender und Empfänger in kurzen Zeitabständen den Kanal und passen die Menge der zu nutzenden Kanäle (Channel Map) an, indem sie dynmaisch verfolgen in welchen Kanälen häufiger Kollisionen auftreten.
% TODO QUELLE https://www.bluetooth.com/blog/how-bluetooth-technology-uses-adaptive-frequency-hopping-to-overcome-packet-interference/