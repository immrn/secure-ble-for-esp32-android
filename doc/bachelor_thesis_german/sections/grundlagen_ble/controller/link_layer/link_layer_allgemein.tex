Der Link Layer nutzt ein gemeinsames Paketformat für das Übertragen von Advertising-Paketen und Anwenderdaten-Pakten, dass in der Abbildung X
% TODO BILD VERWEIS paketformat
dargestellt ist.
% TODO BILD paketformat Spec. 4.0 S. 2200

Die Preamble hat eine Größe von acht Bit und wird genutzt, um auf Empfängerseite die Frequenz zu synchronisieren, die Zeiteinteilung der Symbole zu schätzen und um die automatische Verstärkungskontrolle zu trainieren. Die Preamble beträgt immer 0b01010101, falls das Bit mit dem niedrigsten Stellenwert (LSB für Least Significant Bit) der Access Address 1 ist. Anderenfalls beträgt die Preamble 0b10101010.

Die Access Address hat eine Größe von 32 Bit und identifziert eine Verbindung über den Link Layer bzw. dient dazu Pakete mittels des festgelegten Wertes 0x8E89BED6 als Advertisement-Pakete zu identifzieren. Bevor ein Initiator eine Verbindung zu einem Advertiser aufbaut, erstellt er eine zufällige Access Address, die neben weiteren Bedingungen nicht der des Advertisement-Pakets gleicht oder sich von dieser um ein Bit unterscheidet. Diese Access Address sendet er dann innerhalb der Verbindungsanfrage an den Advertiser.

Die Protocol Data Unit (PDU) unterscheidet sich in die Advertising PDU und Data Channel PDU.
% TODO QUELLE Spec. 4.0 S. 2200 f.

% TODO
% Paketstruktur (S. 2200)
%		PDU (kurz was zur Adv. PDU (adv., scan, init) S. 2201, Data Channel PDU),
%		CRC mit bit stream processing (S. 2216)
% Channel Map S. 2244, Frequnecy hopiing hier? wahrscheinlich