Über dem Physical Layer ist der Link Layer angesiedelt, beginnend mit dem Logical Transport, der in die zwei Betriebsmodi LE Asynchronous Connection (LE ACL) und LE Advertising Broadcast (ADVB) unterteilt.
\\\\
Die LE ACL transportiert Kontrollsignale des über ihr befindlichen Logical Link und Logical Link Control and Adaption Protocol (L2CAP). Außerdem überträgt die LE ACL asynchrone Anwendungsdaten nach dem Best"=Effort"=Prinzip.
\\\\
Mithilfe der Next Expected Sequence Number bzw. Sequence Number (NESN/SN), die jeweils nur die Größe eines Bits besitzen, wird eine einfache Zuverlässigkeit gewährleistet. Empfängt ein Gerät ein Paket B, vergleicht es dessen NESN mit der SN, die es innerhalb des vorherigen Pakets A abgesendet hat. Wenn diese unterschiedlich sind, wurde das vorherige Paket A vom Gegenüber vollständig und korrekt empfangen (ACK für Acknowledgement). Anderenfalls sind die Nummern gleich, was bedeutet, dass das vorherige Paket A nicht vollständig bzw. korrekt empfangen wurde (NACK/NAK für Negative Acknowledgement) und nun erneut an das Gegenüber gesendet werden muss. Zudem prüft das Gerät die SN des empfangenen Pakets B mit der NESN seines vorher gesendeten Pakets A. Sind diese Nummern gleich, wurde das empfangene Paket B vom Gerät erwartet. Anderenfalls sind die Nummern verschieden und somit wurde das Paket B nicht erwartet, weswegen es ignoriert wird. Ein ausführliches Beispiel ist im Anhang \ref{sec: anhang nesn sn} (S. \pageref{sec: anhang nesn sn}) angegeben. Folglich wird auch die Flusskontrolle ermöglicht, da ein Empfänger bei nicht ausreichend freiem Speicher im Buffer ein NACK mithilfe der Sequenznummern zurücksenden kann. \cite{BtSpec4.0_2239-2241}
% QUELLE evtl mit BILD: NESN/SN 4.0 PDF S. 2239 f.
\\\\
Wenn ein Gerät einem Piconet beitritt, wird zwischem dem Master und dem Slave eine Default LE ACL über einen Active Physical Link gebildet. Die Default LE ACL ist einer Access Address zugeordnet. Wird die Default LE ACL getrennt, werden alle Logical Transports zwischen Master und Slave ebenfalls unterbrochen. Bei einem unerwarteten Synchronisationsverlust zum LE Piconet Physical Channel werden der LE Physical Link und alle LE Logical Transports und LE Logical Links unterbrochen.
\\\\
Der ADVB transportiert ohne Verwendung von Acknowledgements Kontrollsignale und Anwendungsdaten bezüglich des Broadcasts über den darunter gelegenen LE Advertising Broadcast Link. Der Datenverkehr ist überwiegend unidirektional, ausgehend vom Advertiser zu allen in Reichweite befindlichen Scannern. Scanner können eine Anfrage an den Advertiser senden, um weitere Anwendungsdaten über den Broadcast zu empfangen. Initiator können eine Anfrage an den Advertiser senden, um eine LE ACL zu bilden. Aufgrund des Verzichts auf Acknowledgements ist der ADVB unzuverlässig, weswegen Pakete redundant übertragen werden. Sobald ein Gerät mit dem Advertising beginnt, wird ein ADVB erzeugt, der anhand der Adresse des Gerätes identifiziert wird. \cite{BtSpec4.0_174-175}
% QUELLE allg. für paragraph: Specification 4.0 PDF S. 174