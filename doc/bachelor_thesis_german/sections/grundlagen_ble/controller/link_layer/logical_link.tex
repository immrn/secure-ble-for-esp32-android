Ein Logical Link unterscheidet sich je nachdem, ob er auf einem LE ACL Logical Transport oder einem ADVB Logical Transport aufbaut und ob er zur Übertragung von Kontrollsignalen oder Anwenderdaten genutzt wird. Jeder Logical Link wird anhand des Logical Link Identifier (LLID) identifiziert, der zwei Bit innerhalb des PDU-Header des Basisbandpakets einnimmt.
% TODO Bildverweis einfügen Paketstruktur

Der Control Logical Link (LE-C) nutzt den darunter liegenden LE ACL Logical Transport, um Kontrollsignale zwischen Geräten im Piconet zu übertragen.

Der User Asynchronous Logical Link (LE-U) nutzt den darunter liegenden LE ACL Logical Transport, um alle asynchronen Anwenderdaten zu übertragen. Mithilfe eines von zwei LLID-Werten wird ein Paket identifiziert. Um den Header des über dem LE-U liegenden Protokolls L2CAP einfach zu halten, beschreibt ein Wert den Beginn eines L2CAP-Frame (das erste Fragment eines L2CAP-Frame) und der andere Wert die Fortsetzung eines L2CAP-Frame (die folgenden Fragmente des L2CAP-Frame), wodurch eine korrekte Synchronisation bei der Zusammensetzung der Fragmente zu einem L2CAP-Frame garantiert wird. Jedoch muss somit ein L2CAP-Frame vollständig übertragen werden, bevor ein neues übertragen wird.

Der Advertising Broadcast Control Logical Link (ADVB-C) nutzt den darunter liegenden Default ADVB, um Kontrollsignale für Verbindungsanfragen oder Anfragen für weitere Broadcast-Anwenderdaten zu übertragen.

Der Advertising Broadcast User Data Logical Link (ADVB-U) nutzt den darunter liegenden Default ADVB, um verbindungslos und ohne den Gebrauch von LE-U Anwenderdaten als Broadcast zu senden.
%TODO QUELLE paragraph: Specification 4.0, S.176 f.