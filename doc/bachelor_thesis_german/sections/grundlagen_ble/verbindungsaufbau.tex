Um mittels BLE ein Piconet zu bilden, benötigt es einen Advertiser. Auf drei vorgegebenen Frequenzen, den Advertising Channels (siehe X)
% TODO SEKTION VERWEIS grundlagen ble controller physical channel
, sendet dieser Daten, mit denen er sich für andere Geräte bemerkbar macht (Advertisements). Dabei können Advertisements auch genutzt werden, um Nutzdaten zu senden. Jedes Advertisement-Paket beinhaltet eine Bluetooth-Adresse des Senders, welche 48 Bit lang ist.

Geräte, die Daten auf den Advertising Channels empfangen, werden Scanner bzw. Initiator genannt. Auf diesem Weg finden sich die Geräte (Discovering). Der Initiator unterscheidet sich vom Scanner, da er in der Lage ist, sich zu einem Advertiser zu verbinden, von dem er ein Advertisement erhielt, dass das Verbinden zu diesem ermöglicht. Sind zwei Geräte verbunden senden und empfangen sie ihre Pakete auf den Data Channels (siehe X).
% TODO SEKTION VERWEIS grundlagen ble controller physical channel
Verbinden sich zwei Geräte wird der Initiator zum Master und der Advertiser zum Slave.
\\\\
Durch die Anwendung von Zeitmultiplexing senden die Geräte ihre Pakete immer zu festgelegten Zeitpunkten. Dabei ist ein Event ein zeitlicher Abschnitt, in dem zusammenhängende Daten in Form von Paketen gesendet bzw. empfangen werden.
% TODO BILD lies folgenden Satz, siehe auch Spec S. 127 oben
In Abbildung X ist ein Advertising Event dargestellt bei dem ein Advertiser auf allen drei Advertising Channels nacheinander Advertisement-Pakete sendet. Auf dem zweiten Kanal empfängt der Advertiser direkt gefolgt auf sein erstes Advertisement-Paket in diesem Kanal ein Paket eines Scanners, auf welches er mit einem weiteren Advertisement antwortet.
% TODO BILD Spec S. 127 unteres bild
In Abbildung X ist ein Advertising Event dargestellt, bei dem ein Initiator auf das Advertisement-Paket eines Advertiser antwortet, um eine Verbindung aufzubauen. Darauf folgt ein Connection Event bei dem Master (ursprünglich Initiator) und Slave (ursprünglich Advertiser) auf einem Data Channel sich gegenseitig Pakete senden. Danach folgt ein weiteres Connection Event auf einem anderen Data Channel.