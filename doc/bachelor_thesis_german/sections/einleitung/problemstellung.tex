Ziel der Arbeit ist zunächst die Untersuchung der Schwachstellen innerhalb von \textit{Bluetooth Low Energy}. Dazu muss ein tieferes Verständnis für dessen Funktionsweise und die zur Verfügung gestellten Sicherheitsfunktionen gewonnen werden. Auf diesem Wissen aufbauend wird eine Infrastruktur entworfen, durch die ein Smartphone und ein Mikrocontroller mittels \textit{Bluetooth Low Energy} sicher miteinander kommunizieren können. Weder bei der Übertragung noch bei der Verarbeitung innerhalb des Systems soll eine außenstehende Instanz auf die übertragenen Nachrichten zugreifen können. Demnach ist eine Lösung für die Probleme der Authentizität, Datenintegrität und Vertraulichkeit erforderlich. Im Idealfall ist die Infrastruktur nicht nur auf die Geräte Smartphone und Mikrocontroller beschränkt, sondern wäre für jede Art von Bluetooth-Gerät denkbar.
\\\\
Desweiteren ist es Ziel der Arbeit, mithilfe dieser Infrastruktur eine Implementierung umzusetzen, die sich auf das Projekt \textit{SteigtUM} bezieht. Da bei dem vorgestellten Verleihdienst sensible Daten zwischen Smartphone und Fahrzeug ausgetauscht werden, wird die Infrastruktur als Grundlage für die sichere Kommunikation genutzt. Dennoch sind weitere Lösungen gefordert. Nur der Nutzer darf Zugriff auf die Funktionen des ausgeliehenen Fahrzeugs erlangen. Demnach muss zu Beginn eines Ausleihprozesses sichergestellt werden, dass der Nutzer berechtigt ist, das Fahrzeug auszuleihen. Verlässt er das Fahrzeug und möchte es nach einer bestimmten Zeit wieder nutzen, soll kein anderer in der Lage sein, das Fahrzeug in der Zwischenzeit zu entsperren und zu entwenden.