\textit{Bluetooth} ist eine weit verbreitete Kommunikationstechnologie, die von vielen Geräten wie Smartphones, Laptops und eingebetteten Systemen unterstützt wird. Seit 2010 unterteilt sich \textit{Bluetooth} in die Varianten \textit{Bluetooth Classic} und \textit{Bluetooth Low Energy}. Zweiteres ist eine geeignete Alternative für Geräte ist, die einen niedrigen Energieverbrauch anstreben.
\\\\
Viele Infrastrukturen, die auf \textit{Bluetooth} basieren, transportieren sensible Daten. Bekannte Anwendungen sind Smartwatches die mit Smartphones kommunizieren oder die Vernetzung von eingebetteten Systemen in der Industrie. Ein weiterer Anwendungsfall sind autonome Verleihdienste für Kleinfahrzeuge innerhalb von Städten. So können Personen Fahrräder oder elektrische Roller ausleihen, um flexibel am Individualverkehr teilzunehmen. Einige dieser Dienste verwenden dabei \textit{Bluetooth}, um zwischen dem Fahrzeug und dem Smartphone des Nutzers Daten zu übertragen. Für diese Anwendungen, stellt sich die Frage, welche Sicherheiten \textit{Bluetooth} bietet und welche Bedingungen dafür erfüllt sein müssen. 
\\\\
Hintergrund dieser Arbeit ist das Projekt \textit{SteigtUM}, das einen solchen autonomen Verleih umsetzen soll. Es unterscheidet sich von anderen Verleihdiensten durch die Nutzung von Lastenfahrrädern. Dabei soll ein neuer Anwendungsfall berücksichtigt werden, der es dem Nutzer ermöglicht das Fahrrad abzustellen und zu einem späteren Zeitpunkt wiederzuverwenden. Auf diese Weise kann der Nutzer bspw. zu einem Geschäfft gelangen, Einkäufe tätigen und diese anschließend nach Hause transportieren. Im Rahmen dieser Arbeit wird die Idee vefolgt, dass das Smartphone des Nutzers mittels \textit{Bluetooth Low Energy} mit dem Mikrocontroller des Fahrzeugs kommuniziert.