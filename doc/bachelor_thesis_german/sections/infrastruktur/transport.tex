Eine der grundlegensten Bedingungen dieser Arbeit ist, dass BLE als Technologie zum übertragen der Daten zwischen Smartphone und Mikrocontroller genutzt werden soll. Demnach stellt sich die Frage nach einem geeigneten Transport der Daten innerhalb des BLE-Protokollstapels.

Es existieren mehrere Referenzmodelle für Kommunikation innerhalb von Rechnernetzen. Geläufig sind das TCP/IP-Referenzmodell (Transport Control Protocol / Internet Protocol), das OSI-Referenzmodell (Open Systems Interconnection) und ein hybrides Referenzmodell aus diesen beiden. Alle drei setzen unterschiedliche Anzahlen von Schichten voraus, von denen sich einige gleichen oder ähneln und andere nicht. Die Eigenschaften der Transportschicht sind bei den drei Referenzmodellen identisch.

Der Protokollstapel des BLE-Controller besteht aus dem Physical Layer, der die Übertragung der Daten auf physischer Ebene definiert, und dem Link Layer, der Pakete

segmente, ports, verbindungslos/verbindungsorientiert, Flusskontrolle, verlustfreie übertragung/sicherstellung, Reihenfolge
% TODO QUELLE https://katalog.ub.tu-freiberg.de/Record/0-1666728691, https://link.springer.com/book/10.1007%2F978-3-658-26356-0, Computer Networks / Computernetze von Christian Baun, S. 39 Transportschicht