Die Infrastruktur beschreibt eine allgemeine Lösung, um eine sichere Kommunikation mittels Bluetooth Low Energy (BLE) zwischen einem Mikrocontroller und einem Smartphone zu gewährleisten. Bis auf den Fakt, dass ein Mikrocontroller und ein Smartphone über BLE miteinander kommunizieren, sind die spezifischen Systeme der beiden Kommunikationsparteien nicht vorgegeben. Dennoch ist die Lösung universell einsetzbar, solange jedes System über ein Bluetooth"=Modul (mind. der Bluetooth"=Version 4.0) verfügt. Da das Protokoll \textit{Transport Layer Security} (TLS) ein wesentlicher Bestandteil des Lösungsansatzes ist, müssen die Systeme eine Software"=Bibliothek für TLS unterstützen. Somit ist die Infrastruktur unabhängig von dem in der Sektion \ref{sec: einleitung} beschriebenen Projekt \textit{SteigtUM}.
\\\\
Da das Ziel dieser Infrastruktur die sichere Datenübertragung zwischen zwei Kommunikationsparteien ist, sollte diesbezüglich "`Sicherheit"' definiert werden. Nach \cite{Bless2005_19-20} werden mehrere Sicherheitsziele definiert um ein sicheres Netzwerk zu schaffen. Daraus ergeben sich die für diese Infrastruktur wesentlichen Sicherheitsziele der Vertraulichkeit, Datenintegrität und Authentizität.
% QUELLE S. 19 https://books.google.de/books?id=-fciBAAAQBAJ&lpg=PA2&ots=YPh7AK5_WJ&dq=sichere%20Daten%C3%BCbertragung&lr&pg=PA19#v=onepage&q&f=false

Vertraulichkeit bedeutet, dass "`Übertragene Daten [...] nur berechtigten Instanzen zugänglich sein [sollen], d.h. keine unbefugte dritte Partei soll an den Inhalt von übertragenen Nachrichten gelangen können"' \cite{Bless2005_19}.
% QUELLE ZITAT
Die Datenintegrität trägt folgende Bedeutung. "`Für den Empfänger muss eindeutig erkennbar sein, ob Daten während ihrer Übertragung unbefugt geändert wurden"' \cite{Bless2005_19}.
% QUELLE ZITAT 
Authentizität teilt sich in zwei Punkte auf. Zum einen soll "`eine Instanz [...] einer anderen ihre Identität zweifelsfrei nachweisen können (Identitätsnachweis bzw. Authentifizierung der Instanz)"' \cite{Bless2005_19}. 
% QUELLE ZITAT
Zum anderen "`soll überprüft werden können, ob eine Nachricht von einer bestimmten Instanz stammt (Authentizität der Daten)"' \cite{Bless2005_19}.
% QUELLE ZITAT