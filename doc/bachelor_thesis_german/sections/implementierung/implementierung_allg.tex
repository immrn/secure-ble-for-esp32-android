Mithilfe der definierten Infrastruktur kann nun eine Implementierung für einen spezifischen Anwendungsfall erstellt werden. In dieser Arbeit wird die Infrastruktur für einen autonomen Verleih von elektrischen Kleinfahrzeugen in Bezug auf das Projekt \textit{SteigtUM} (siehe Sektion \ref{sec: einleitung}) angewandt. Der Verleihdienst sieht vor, dass ein Benutzer mit einem Smartphone ein Kleinfahrzeug ausleihen kann. In Bezug auf diese Arbeit sollen Smartphone und Kleinfahrzeug (Mikrocontroller) mittels BLE miteinander kommunizieren. Um einen Ausleihprozess einzuleiten wird ein Back End in Form eines Servers benötigt, damit ein Verwaltungsmodell bzgl. Zahlungen realisiert werden kann. Dieses Modell ist jedoch nicht Bestandteil der Arbeit. Dennoch wird für den Prototyp (und für die spätere reale Implementierung) ein Back End benötigt, um den Benutzer bzw. die Smartphone-Anwendung (App) zum Ausleihen eines Fahrzeugs zu autorisieren.
\\\\
Innerhalb des Prototyps kommunizieren anstelle eines Smartphones und eines Mikrocontrollers vorerst nur zwei Mikrocontroller. Ein Mikrocontroller wird mit dem Verleihfahrzeug assoziiert, während der andere das Smartphone des Benutzers vertritt. Die weiteren Kommunikationsparteien Back End und Zertifizierungsstelle werden nur simuliert.
\\\\
Für die Implementierung wurden einige Bedingungen aufgestellt, damit sie das konkrete Modell eines Verleihdienstes nicht einschränkt. Eine Bedingung ist, dass zu Beginn eines Ausleihprozesses die App mit dem Back End (sicher) kommunizieren kann, bis der Benutzer das Fahrzeug nutzen kann. Danach darf die Implementierung keine Verbindung zwischen App und Back End weiterhin voraussetzen. Zwischen dem Fahrzeug und dem Back End darf nie eine Verbindung vorausgesetzt werden. Wird die Bluetooth-Verbindung zwischen App und Fahrzeug unterbrochen, darf nicht vorausgesetzt werden, dass zwischen App und Back End eine (sichere) Verbindung hergestellt werden kann. Somit bleibt die Implementierung weitestgehend unabhängig von äußeren Einflüssen, die die Verbindungen zum Back End beeinträchtigen können. Außerdem gibt die Implementierung nicht vor, welche technischen Voraussetzungen das Fahrzeug für die Kommunikation benötigt (mit Ausnahme von BLE).