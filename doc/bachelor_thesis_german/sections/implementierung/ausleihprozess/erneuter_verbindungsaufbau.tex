Ab diesen Punkt muss ein weiterer Fall für den Verlauf des Ausleihprozesses berücksichtigt werden: der Abbruch der Verbindung zwischen Smartphone und Fahrzeug sowie die darauffolgende Wiederherstellung der Verbindung. Der Verbindungsabbruch kann zum einen durch technische Einflüsse (z.B. Interferenzen) verursacht werden und nicht vom Benutzer beabsichtigt sein. Zum anderen ist es denkbar, dass der Nutzer sich für eine bestimmte Zeitspanne vom Fahrzeug entfernt, wodurch die Verbindung aufgrund der begrenzten Bluetooth-Reichweite abbricht. Evtl. sollte hier der Benutzer dem Fahrzeug signalisieren, dass er für eine bestimmte Zeitspanne abwesend ist. Danach begibt sich der Nutzer wieder zum Fahzeug und möchte es weiter nutzen.
\\\\
Wenn die Verbindung nun abbricht, beginnt das Fahrzeug mit dem Advertising. Sobald der Benutzer bzw. das Smartphone in Reichweite ist, verbindet sich die App per BLE mit dem Fahrzeug, da es die Bluetooth-Adresse des Fahrzeugs bereits kennt, und baut wie oben beschrieben eine sichere Verbindung auf. Nun sendet die App die alte Subscription erneut an das Fahrzeug. Das Fahrzeug prüft erneut die Subscription und vergleicht sie mit der Subscription, die es zu Beginn des Ausleihprozesses empfangen hat. Wurde die Subscription erfolgreich verifiziert und stimmt sie mit der ursprünglichen Subscription überein, kann das Fahrzeug wieder genutzt werden.