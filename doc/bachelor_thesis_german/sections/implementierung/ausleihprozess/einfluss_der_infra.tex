Da die Zertifikate der Parteien (Back End, App, Fahrzeug) von der Zertifizierungsstelle des Verleihdienstes ausgestellt wurden und die Parteien nur das Root-Zertifikat des Verleihdienstes nutzen, um das Zertifikat eines Gegenüber zu verifizieren, können nur die Parteien TLS-Verbindungen zueinander aufbauen. Jede außenstehende Instanz, die versucht eine Verbindung zu einer der Parteien (Back End, App, Fahrzeug) aufzubauen, wird bereits beim TLS-Handshake abgewiesen. Grund dafür ist, dass die außenstehende Instanz kein von der Zertifizierungsstelle ausgestelltes Zertifikat besitzt. Sollte sie doch an ein solches Zertifikat gelangen, muss sie beim TLS-Handshake immer noch beweisen, dass sie den zugehörigen privaten Schlüssel besitzt. Aus diesem Grund muss jede der Parteien (Back End, App, Fahrzeug) den privaten Schlüssel geheim halten. Das bedeutet auch, dass ein Angreifer nicht in der Lage sein darf, den privaten Schlüssel aus einer der Parteien auszulesen (z.B. bei physischen Zugang). Deshalb sollten Zertifikate und private Schlüssel in Keystores gespeichert werden.