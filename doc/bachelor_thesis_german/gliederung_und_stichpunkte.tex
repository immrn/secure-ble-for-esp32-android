\documentclass[doktyp=barbeit]{TUBAFarbeiten}

\usepackage{selinput}% Auswahl der Dateikodierung (ansi,latin1,utf8,...)
	\SelectInputMappings{adieresis={ä},germandbls={ß},Euro={€}}% Zeichenzuordnung für selinput.sty
\usepackage[T1]{fontenc}% Einstellung Fontencoding

\usepackage{csquotes}% Einstellung zu Anführungszeichen; wird von biblatex.sty gefordert
\usepackage[backend=biber,sortlocale=de_DE_phonebook]{biblatex}% bessere Literaturverarbeitung
\addbibresource{tubafarbeiten-beispiel.bib}% welche Literaturdatenbank genutzt werden soll; Endung nicht vergessen!

%\usepackage{setspace}% Einstellungen Zeilenabstand
	%\onehalfspacing% Einstellungen Zeilenabstand

\TUBAFFakultaet{Fakultät für Mathematik und Informatik}
\TUBAFInstitut{Institut für Informatik}
\TUBAFLehrstuhl{Lehrstuhl für Betriebssysteme und Kommunikationstechnologien}

%\TUBAFZweitlogo{\includegraphics{thekla_logo.jpg}}

\TUBAFTitel[Entwicklung einer Sicherheitsinfrastruktur zur Bluetooth-Kommunikation zwischen Smartphone und Mikrocontroller]{Entwicklung einer Sicherheitsinfrastruktur zur Bluetooth-Kommunikation zwischen Smartphone und Mikrocontroller}
\TUBAFBetreuer{Prof.\,Dr. Konrad Froitzheim}
\TUBAFKorrektor{M.Sc. Jonas Treumer}%TODO Betreuer Lorenzo Neumann
\TUBAFAutor[M. Käsemodel]{Marian Käsemodel}
\TUBAFStudiengang{Angewandte Informatik}
\TUBAFVertiefung{Technik}
\TUBAFMatrikel{62\,412}
\TUBAFDatum[]{\today}%TODO

\begin{document}

\maketitle

\TUBAFErklaerungsseite


\KOMAoptions{
	listof=totoc	% Abbildungs- und Tabellenverzeichnis im Inhaltsverzeichnis
}

\tableofcontents
\listoffigures
\listoftables

\section{Einleitung}
\subsection{Themenstellung}
\begin{itemize}
	\item BT als Kommunikationstechnolgie verbreitet -> BLE als Variante mit geringem Energieverbrauch ideal für batteriebetriebene Systeme -> überleiten zu SteigtUM
	\item Projekt SteigtUM
	\begin{itemize}
		\item ist Hintergrund dieser Arbeit, erklären: Verleihdienst für elektrische Kleinfahrzeuge (u.a. Lastenfahrräder)
		\item Individualverkehr für Kurzstrecken emissionsfrei gestalten
		\item Kommunikation muss sicher sein, da Buchungsvorgänge, Datenschutz, verhindern von Manipulation, ...
	\end{itemize}
	\item daraus auf allgemeine Lösung ableiten für eine sichere Infrastruktur basierend auf BLE zwischen MCU und Smartphone
	\item Infrastruktur beschränkt sich nicht zwangsweise nur auf MCU und Smartphone
\end{itemize}
\subsection{Problemstellung}
\begin{itemize}
	\item Ziel ist Entwicklung einer Infrastruktur, die eine sichere Kommunikation zwischen MCU und Smartphone ermöglicht
	\item dabei sollen MCU und Smartphone nur mittels BLE miteinander kommunizieren
	\item BLE weißt einige Schwachstellen auf, weswegen eine Lösung gefordert ist um BLE sicher zwischen MCU und Smartphone zu nutzen
	\begin{itemize}
		\item Schutz vor Außenstehenden und Software, die auf dem Smartphone agiert.
	\end{itemize}
	\item Zum Beweis der Funktionsfähigkeit der Infrastruktur dient eine Implementierung, die sich auf den Ausleihprozess des Projektes SteigtUM bezieht
	\begin{itemize}
		\item weitere Kommunikationsparteien (Backend) und deren Verbindungen werden nur simuliert dargestellt
		\item Verbindung zwischen Fahrrad und Backend nicht zwingend erforderlich
		\item bei Wiederherstellung einer abgebrochenen Verbindung soll die Verbindung zwische App und Backend nicht zwingend sein
	\end{itemize}
\end{itemize}

\section{Grundlagen zu Bluetooth Low Energy}
\subsection{Abgrenzung von Bluetooth Classic}
\begin{itemize}
	\item Wesentliche Unterschiede zwischen BLE und BT Classic
\end{itemize}
\subsection{Aufbau}
	\subsubsection{Controller Stack}
	\begin{itemize}
		\item Funktionsweise der Layer: PHY und LL (Link Layer)
	\end{itemize}
	\subsubsection{Host Controller Interface}
	\subsubsection{Host Stack}
	\begin{itemize}
		\item Funktionsweise der Layer: L2CAP (Logical Link Control and Adaption Protocol), GATT (Genric Attribute Profile), GAP (Generic Access Profile), Argumentation, wieso L2CAP (und nicht GATT) genutzt wird erst in Sektion "Infrastruktur"
		\item SMP (Security Manager Protocol)
	\end{itemize}
\subsection{Verbindungsaufbau / Pairing}
\begin{itemize}
	\item Pairing-Methoden vorstellen mit Unterschieden in den BT-Versionen
\end{itemize}
\subsection{Sicherheitslücken}
\begin{itemize}
	\item Pairing-Methoden nicht in allen Versionen sicher (MITM, passives Abhören)
	\item Keine Application-Layer-Security
	\item ...
\end{itemize}

\section{Grundlagen zu Transport Layer Security}
\begin{itemize}
	\item allgemeine Erläuterung
	\item sichere Versionen
	\item nicht nur für gewöhnliche Anwendungsbeispiele geeignet
\end{itemize}
\subsection{Zertifikate}
\begin{itemize}
	\item Aufbau
	\item CAs
	\item Authentifikation/Authentifizierung
\end{itemize}
\subsection{Algorithmen}
\begin{itemize}
	\item Schlüsselaustausch
	\item Verschlüsselung
	\item Datenintegrität
	\item Ciphersuits
\end{itemize}
\subsection{Protokoll}
\begin{itemize}
	\item Handshake
	\item Record
\end{itemize}
\subsection{Sicherheit}
\begin{itemize}
	\item Angriffe gegen TLS
	\item Forward Secrecy
	\item TLS Interception
\end{itemize}

\section{Infrastruktur}
\subsection{Ziel der Infrastruktur}
\begin{itemize}
	\item sichere Kommunikation zwischen Smartphone und MCU mittels TLS über BLE
	\item ist unabhängig vom Projekt SteigtUM
\end{itemize}
\subsection{Topologie}
\begin{itemize}
	\item Client, Server
	\item CA
\end{itemize}
\subsection{Transport}
\begin{itemize}
	\item Verwendung von L2CAP (und warum GATT ungeeignet ist)
\end{itemize}
\subsection{Sicherheit}
\begin{itemize}
	\item Positionierung und Begründung, dass Sicherheitsfeatures von BLE keinen vollständigen Schutz bieten (auch keine App-Layer-Security)
	\item deswegen TLS verwenden für MITM-Protection, Schutz gegen passives Anhören, Datenintegrität, Ende-zu-Ende-Verschlüsselung, App-Layer-Security
\end{itemize}
\subsection{Verbindungsaufbau}
\begin{itemize}
	\item GAP (gewählte Pairing-Methode (nach BT-Version) dürfte keinen Einfluss haben)
	\item TLS-Handshake
	\item erst in Sektion "Implementierung" auf Subscription eingehen
\end{itemize}

\section{Implementierung}
\subsection{Ziel der Implementierung}
\begin{itemize}
	\item Bezug zu SteigtUM / Verleih elektrischer Kleinfahrzeuge
	\item Subscription-Modell sorgt dafür, dass das Fahrzeug sicherstellen kann, dass es vom Nutzer ausgeliehen werden darf
\end{itemize}
\subsection{Topologie}
\begin{itemize}
	\item Client
	\item Server
	\item Backend
	\item CA
\end{itemize}
\subsection{Hardware und Software}
\begin{itemize}
	\item Android Smartphone, Android- und BT-Version
	\item Android Bibliotheken, ...
	\item MCU: ESP32... , BT-Version, weitere Eigenschaften/Features
	\item ESP Software: FreeRTOS, SPIFFS (Filesystem), nimBLE, mbedTLS (TLS Version angeben)
\end{itemize}
\subsection{Transport und Sicherheit}
\begin{itemize}
	\item L2CAP
	\item Verwaltung des RX-Buffers
	\item Durchsatz
	\item Konfiguration (MTU, ...)
	\item gewählte Ciphersuites
	\item ...
\end{itemize}
\subsection{Ausleihprozess}
\begin{itemize}
	\item Ausgangssituationen, Ablauf und Probleme des Ausleihprozesses
\end{itemize}
	\subsubsection{Verbindungsaufbau}
	\begin{itemize}
		\item ähnlich wie bei Erklärung der Infrastruktur nur mit Subscription und deren Übertragung und Verifizierung
		\item erläutern was passiert, wenn Verbindung zum Fahrzeug abgebrochen ist (bspw. durch Abstellen des Fahrzeugs und kurzzeitiges Verlassen der BLE-Funkreichweite) und nun wiederhergestellt werden soll
	\end{itemize}
	\subsubsection{Beenden des Ausleihprozesses}
	\begin{itemize}
		\item erklären, wie sichergestellt wird, dass der Ausleihprozess vom Nutzer aus beendet wurde
		\item bzw. Beendigung der Verbindung durch das Fahrzeug, wenn die Standzeit (Nutzer verließ BLE-Funkreichweite) überschritten wurde
	\end{itemize}

\section{Ausblick}
\begin{itemize}
	\item Weiterführung der Arbeit
\end{itemize}

\section{Zusammenfassung}

\printbibliography[heading=bibintoc]

\end{document}